%% This is file `elsarticle-template-1-num.tex',
%%
%% Copyright 2009 Elsevier Ltd
%%
%% This file is part of the 'Elsarticle Bundle'.
%% ---------------------------------------------
%%
%% It may be distributed under the conditions of the LaTeX Project Public
%% License, either version 1.2 of this license or (at your option) any
%% later version.  The latest version of this license is in
%%    http://www.latex-project.org/lppl.txt
%% and version 1.2 or later is part of all distributions of LaTeX
%% version 1999/12/01 or later.
%%
%% Template article for Elsevier's document class `elsarticle'
%% with numbered style bibliographic references
%%
%% $Id: elsarticle-template-1-num.tex 149 2009-10-08 05:01:15Z rishi $
%% $URL: http://lenova.river-valley.com/svn/elsbst/trunk/elsarticle-template-1-num.tex $
%%
\documentclass[preprint,12pt]{elsarticle}

%% Use the option review to obtain double line spacing
%% \documentclass[preprint,review,12pt]{elsarticle}

%% Use the options 1p,twocolumn; 3p; 3p,twocolumn; 5p; or 5p,twocolumn
%% for a journal layout:
%% \documentclass[final,1p,times]{elsarticle}
%% \documentclass[final,1p,times,twocolumn]{elsarticle}
%% \documentclass[final,3p,times]{elsarticle}
%% \documentclass[final,3p,times,twocolumn]{elsarticle}
%% \documentclass[final,5p,times]{elsarticle}
%% \documentclass[final,5p,times,twocolumn]{elsarticle}

%% The graphicx package provides the includegraphics command.
\usepackage{graphicx}
%% The amssymb package provides various useful mathematical symbols
\usepackage{amssymb}
%% The amsthm package provides extended theorem environments
%% \usepackage{amsthm}

%% The lineno packages adds line numbers. Start line numbering with
%% \begin{linenumbers}, end it with \end{linenumbers}. Or switch it on
%% for the whole article with \linenumbers after \end{frontmatter}.
\usepackage{lineno}

%% natbib.sty is loaded by default. However, natbib options can be
%% provided with \biboptions{...} command. Following options are
%% valid:

%%   round  -  round parentheses are used (default)
%%   square -  square brackets are used   [option]
%%   curly  -  curly braces are used      {option}
%%   angle  -  angle brackets are used    <option>
%%   semicolon  -  multiple citations separated by semi-colon
%%   colon  - same as semicolon, an earlier confusion
%%   comma  -  separated by comma
%%   numbers-  selects numerical citations
%%   super  -  numerical citations as superscripts
%%   sort   -  sorts multiple citations according to order in ref. list
%%   sort&compress   -  like sort, but also compresses numerical citations
%%   compress - compresses without sorting
%%
%% \biboptions{comma,round}

% \biboptions{}

\journal{Journal Name}

\begin{document}

\begin{frontmatter}

%% Title, authors and addresses

\title{Template of Final Report NDVW \footnote{1) See appendix. 2) Report's license CC, Software's license: GPL, PLEASE add here url with source code or demo of the project} }


%% use the tnoteref command within \title for footnotes;
%% use the tnotetext command for the associated footnote;
%% use the fnref command within \author or \address for footnotes;
%% use the fntext command for the associated footnote;
%% use the corref command within \author for corresponding author footnotes;
%% use the cortext command for the associated footnote;
%% use the ead command for the email address,
%% and the form \ead[url] for the home page:
%%
%% \title{Title\tnoteref{label1}}
%% \tnotetext[label1]{}
%% \author{Name\corref{cor1}\fnref{label2}}
%% \ead{email address}
%% \ead[url]{home page}
%% \fntext[label2]{}
%% \cortext[cor1]{}
%% \address{Address\fnref{label3}}
%% \fntext[label3]{}


%% use optional labels to link authors explicitly to addresses:
%% \author[label1,label2]{<author name>}
%% \address[label1]{<address>}
%% \address[label2]{<address>}

\author{John Smith}

\address{Spain}

%\begin{abstract}
%% Text of abstract

%\end{abstract}

%\begin{keyword}
%Science \sep Publication \sep Complicated
%% keywords here, in the form: keyword \sep keyword

%% MSC codes here, in the form: \MSC code \sep code
%% or \MSC[2008] code \sep code (2000 is the default)

%\end{keyword}

\end{frontmatter}

%%
%% Start line numbering here if you want
%%
%\linenumbers

%% main text
\section{Introduction}
\label{S:1}

Section to be completed during Sprint 1: see CV task Sprint 1. 

You can focus the project either as a game/VW  with a user/player who interacts with the agents and the environment, or more like a simulation (without user/player) in which agents interact each other and with the environment. Describe here briefly the setting of your game/simulation: landscape, characters, objects, locations (house, cave, ...) of game agents activities, player's goal.

Game/VW:
\begin{itemize}
\item Intro
\item Scenario
\item Game/VW Agent/s (GA)
You should propose the behavior of one or two game agents. The agent should make use of senses (see, touch, smell...), movement, even you can make use of communication between agents. Agents can also have state variables that change and affect either its own behavior or other agents' behavior. In the following items, describe the expected behavior of each game agent.
\item Behaviour of GA (textual description, GA black box)
\item Behaviour of GA ... 
\end{itemize}

Which AI approach have you selected to design GA behaviour?

\section{Related work}

To be completed with the games that use similar approaches (included also in the oral presentation)


\section{Proposal}

\subsection {AI Design}

Details of the AI Design of the proposed behaviour (to be done in sprint 1) 

\subsection{Software architecture}

(to be done between sprint 2 and 3)

\subsection{Another section with details...}

\section{Simulations and Results}

(to be done between sprint 3 and the final upload in january)

\subsection{Methodology}

Details of the Unity scene, characters, animations, used assets

\subsection{Results}

Screenshots and links to video (to be included just before or after Christmas)

Benchmarks

\begin{table}[h]
\centering
\begin{tabular}{l l l}
\hline
\textbf{Treatments} & \textbf{Response 1} & \textbf{Response 2}\\
\hline
Treatment 1 & 0.0003262 & 0.562 \\
Treatment 2 & 0.0015681 & 0.910 \\
Treatment 3 & 0.0009271 & 0.296 \\
\hline
\end{tabular}
\caption{Table caption}
\end{table}

\begin{figure}[h]
\centering\includegraphics[width=0.4\linewidth]{placeholder}
\caption{Figure caption}
\end{figure}


\section{Conclusions and Future work}

Give your conclusions and ideas for future work. 


%% The Appendices part is started with the command \appendix;
%% appendix sections are then done as normal sections
\appendix

\section{Sprint 1, oct. 23rd, 2023}
\label{sprint1}
\subsection{Planned tasks}

Describe planned tasks  and who is the responsible of each task

\subsection{Work Done}

Describe each task and who did it 

\subsection{Work in progress}

Explain tasks in progress 

\section{Sprint 2, nov. 20th, 2023}
\label{sprint2}

\subsection{Planned tasks}

Describe planned tasks  and who is the responsible of each task

\subsection{Work Done}

Describe each task and who did it 

\subsection{Work in progress}

Explain tasks in progress 


\section{Sprint 3, dec. 18th, 2023}
\label{sprint3}

\subsection{Planned tasks}

Describe planned tasks  and who is the responsible of each task

\subsection{Work Done}

Describe each task and who did it 

\subsection{Work in progress}

Explain tasks in progress \\


%% References
%%
%% Following citation commands can be used in the body text:
%% Usage of \cite is as follows:
%%   \cite{key}          ==>>  [#]
%%   \cite[chap. 2]{key} ==>>  [#, chap. 2]
%%   \citet{key}         ==>>  Author [#]

%% References with bibTeX database:

\bibliographystyle{model1-num-names}
\bibliography{sample.bib}

%% Authors are advised to submit their bibtex database files. They are
%% requested to list a bibtex style file in the manuscript if they do
%% not want to use model1-num-names.bst.

%% References without bibTeX database:

% \begin{thebibliography}{00}

%% \bibitem must have the following form:
%%   \bibitem{key}...
%%

% \bibitem{}

% \end{thebibliography}


\end{document}

%%
%% End of file `elsarticle-template-1-num.tex'.