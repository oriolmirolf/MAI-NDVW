%% This is file `elsarticle-template-1-num.tex',
%%
%% Copyright 2009 Elsevier Ltd
%%
%% This file is part of the 'Elsarticle Bundle'.
%% ---------------------------------------------
%%
%% It may be distributed under the conditions of the LaTeX Project Public
%% License, either version 1.2 of this license or (at your option) any
%% later version.  The latest version of this license is in
%%    http://www.latex-project.org/lppl.txt
%% and version 1.2 or later is part of all distributions of LaTeX
%% version 1999/12/01 or later.
%%
%% Template article for Elsevier's document class `elsarticle'
%% with numbered style bibliographic references
%%
%% $Id: elsarticle-template-1-num.tex 149 2009-10-08 05:01:15Z rishi $
%% $URL: http://lenova.river-valley.com/svn/elsbst/trunk/elsarticle-template-1-num.tex $
%%
\documentclass[preprint,12pt]{elsarticle}

%% Use the option review to obtain double line spacing
%% \documentclass[preprint,review,12pt]{elsarticle}

%% Use the options 1p,twocolumn; 3p; 3p,twocolumn; 5p; or 5p,twocolumn
%% for a journal layout:
%% \documentclass[final,1p,times]{elsarticle}
%% \documentclass[final,1p,times,twocolumn]{elsarticle}
%% \documentclass[final,3p,times]{elsarticle}
%% \documentclass[final,3p,times,twocolumn]{elsarticle}
%% \documentclass[final,5p,times]{elsarticle}
%% \documentclass[final,5p,times,twocolumn]{elsarticle}

%% The graphicx package provides the includegraphics command.
\usepackage{graphicx}
%% The amssymb package provides various useful mathematical symbols
\usepackage{amssymb}
\usepackage{float}
\usepackage{amsmath}
%% The amsthm package provides extended theorem environments
%% \usepackage{amsthm}

%% The lineno packages adds line numbers. Start line numbering with
%% \begin{linenumbers}, end it with \end{linenumbers}. Or switch it on
%% for the whole article with \linenumbers after \end{frontmatter}.
\usepackage{lineno}
\usepackage{hyperref}
\hypersetup{
    colorlinks=true,
    linkcolor=blue,
    citecolor=blue,
    urlcolor=blue,
}
\usepackage{multimedia}
\usepackage{attachfile2}
\usepackage{tikz}
\usetikzlibrary{shapes,arrows,positioning,fit}
\usepackage{listings}
\usepackage{xcolor}
\lstset{
  basicstyle=\ttfamily\small,
  frame=single,
  breaklines=false,
  columns=fullflexible,
  keepspaces=true
}

% Define JSON language for listings
\lstdefinelanguage{json}{
  basicstyle=\ttfamily\small,
  numbers=none,
  numberstyle=\scriptsize,
  stepnumber=1,
  numbersep=8pt,
  showstringspaces=false,
  breaklines=true,
  frame=lines,
  string=[s]{"}{"},
  comment=[l]{:\ "},
  morecomment=[l]{:"},
  literate=
    *{0}{{{\color{blue}0}}}{1}
     {1}{{{\color{blue}1}}}{1}
     {2}{{{\color{blue}2}}}{1}
     {3}{{{\color{blue}3}}}{1}
     {4}{{{\color{blue}4}}}{1}
     {5}{{{\color{blue}5}}}{1}
     {6}{{{\color{blue}6}}}{1}
     {7}{{{\color{blue}7}}}{1}
     {8}{{{\color{blue}8}}}{1}
     {9}{{{\color{blue}9}}}{1}
}

%% natbib.sty is loaded by default. However, natbib options Fcan be
%% provided with \biboptions{...} command. Following options are
%% valid:

%%   round  -  round parentheses are used (default)
%%   square -  square brackets are used   [option]
%%   curly  -  curly braces are used      {option}
%%   angle  -  angle brackets are used    <option>
%%   semicolon  -  multiple citations separated by semi-colon
%%   colon  - same as semicolon, an earlier confusion
%%   comma  -  separated by comma
%%   numbers-  selects numerical citations
%%   super  -  numerical citations as superscripts
%%   sort   -  sorts multiple citations according to order in ref. list
%%   sort&compress   -  like sort, but also compresses numerical citations
%%   compress - compresses without sorting
%%
%% \biboptions{comma,round}

% \biboptions{}

\journal{NDVW Sprint 3}

\begin{document}


\begin{titlepage}
    \centering
    \vspace*{1cm}

    % University Logo
    \includegraphics[width=0.3\textwidth]{figures/ub_logo.png}

    \vspace{1.5cm}
    {\Huge\bfseries NDVW Final Report\par}
    \vspace{1cm}
    {\large Procedurally Generated 2D RPG with AI-Driven Narrative\par}
    \vspace{2cm}
    {\large\bfseries Group 3:\par}
    \vspace{0.2cm}
    {\large Oriol MIR\'O, Jean DI\'E, Bruno S\'ANCHEZ, D\'aniel M\'ACSAI\par}
    \vspace{3cm}
    {\large \textbf{Master in Artificial Intelligence}\par}
    \vspace{0.5cm}
    {\large \textbf{Normative and Dynamic Virtual Worlds}\\ Final Report (Sprint 3)\par}
    \vspace{1cm}
    {\large\bfseries December 16th, 2025\par}
\end{titlepage}


%%
%% Start line numbering here if you want
%%
%\linenumbers

%% main text
\section{Introduction}
\label{S:1}

\subsection{General Overview}

The project involves developing a simple 2D top-down game in which a player navigates a procedurally generated dungeon filled with enemies, obstacles, and various NPCs. 
Our goal is to generate almost every element of the game procedurally, from the layout of the dungeon, to the story -- within some bounds, of course --  and even the ambient music.
The main challenge is to make all this generated content consistent, which requires designing a generation pipeline.
Moreover, we designed a story so that it is explained \emph{why} so many things change across runs.
There is three parts to the story, each with a boss at the end. 
Defeating each boss also serves as a checkpoint, and the generated story and dungeon are frozen up to that point. 
See Figure~\ref{fig:game_flow} for a schematic representation of this.

\begin{figure}[H]
    \centering
    \includegraphics[width=\textwidth]{figures/game_flow.png}
    \caption{Three dungeon chapters, each ending in a boss fight. Defeating bosses 1 and 2 seals solves chapter (layout and story become fixed). Upon death, the player respawns at the start of the current unsolved chapter; solved chapters remain unchanged.}
    \label{fig:game_flow}
\end{figure}

\paragraph{Where is the AI in this game?}
We build the baseline gameplay loop (movement, combat, weapons, UI) from the tutorial in Section~\ref{sec:tutorial}. 
On top of that baseline, our project contributions are the AI-driven systems:

\begin{itemize}
    \item \textbf{PCG (Dungeon + population):} 
      BSP-based layout generation + connectivity, plus room interior/population (incl.\ cellular automata) to create different dungeon structures across runs.
    \item \textbf{GenAI (Narrative + Music):} 
      We analyze room screenshots (vision) to produce structured descriptors, then generate NPC dialogue/quests/lore (LLM), and generate adaptive music conditioned on the same descriptors.
    \item \textbf{RL (Boss agents):} 
      Bosses are controlled by policies trained in a simplified combat environment, then deployed in-game to produce reactive, non-scripted encounters.
\end{itemize}

\paragraph{Contributions}
A detailed task allocation (who implemented what) is provided in Appendix~\ref{app:contrib}.

\subsection{Scenario}
\label{sec:scenario}

We now present the story of our game. The goal is to keep the lore tightly aligned with the mechanics: death regenerates content, and boss victories make what happened so far persistent.

Sir Cael is trapped inside a magic book that rewrites the dungeon every time he dies. 
The dungeon is split into three chapters, each one ending in a boss fight. 
When Cael dies, the book rewrites only the chapter he is currently in: rooms and encounters are rearranged into a new version of the same chapter, which explains the procedural regeneration across runs.

Defeating the bosses of Chapter 1 and Chapter 2 \emph{seals} those chapters. 
Once sealed, their layouts and narrative elements stop changing. 
If Cael dies later, the sealed chapters remain intact and only the current, unsealed chapter regenerates -- this justifies the checkpoint logic

In the final chapter, Cael confronts the Chronicler, the force behind the loop. If he wins, the rewriting stops and the story ends.

\subsubsection{Spoilers!}
Cael is the Chronicler. 
Long ago, he created the Chronicle to rewrite history until it was ``perfect'', but he ended up trapped in his own loop.
 Boss 1 and boss 2 are manifestations of what keeps him stuck (fear and denial), and sealing the first two chapters represents accepting those parts instead of rewriting them. 
 The final boss is Cael facing the part of himself that refuses to stop revising the story.


\subsection{Tutorial we start from}\label{sec:tutorial}

To accelerate development of the core mechanics, we base the initial phase of our project on the online tutorial course ``Unity 2D RPG: Complete Combat System'' (Unity 2D RPG: Complete Combat System) from Udemy.
All the information can be found in \href{https://www.udemy.com/course/unity-2d-rpg/?couponCode=MT251110G3}{this link}.
Figure~\ref{fig:ex_fig_tutorial} shows an example scene from the tutorial.
This tutorial guides us through creating a classic 2D top-down RPG in Unity using C\#. 
It covers topics such as player movement, tilemap and rule tile usage, weapon systems, and basic combat.
By using this foundation, we can focus our efforts on the AI side of the game rather than reinventing standard mechanics from scratch.

Specifically, the tutorial includes:
\begin{itemize}
  \item Basic 2D top-down player movement and animation (using tilemaps and rule tiles)  
  \item Implementation of a weapon-based combat system in Unity (including multiple weapons, attack animations, and hit detection)  
  \item Setting up scene workflow and tiled environments using Unity's tilemap features  
  \item Some C\# fundamentals tied to game logic in Unity (moving from beginner to intermediate level)  
\end{itemize}

What the tutorial does \textbf{not} cover (and which our project extends) includes:
\begin{itemize}
  \item Procedural content generation of dungeon layouts (BSP-based layout generator and CA-based room populator)  
  \item Narrative generation tied to player runs and persistent checkpointing of story chapters  
  \item Procedural ambient music generation and dynamic layering based on gameplay state  
  \item Reinforcement learning-based boss behavior trained through self-play  
  \item The novel ``checkpoint/hall of fixed chapters'' mechanic, where defeating bosses 1 and 2 freezes that portion of the dungeon and story  
\end{itemize}

All scenes in the tutorial are manually designed, and all the enemies' movement and attack patterns are random, not tied to any player input.

\begin{figure}
    \centering
    \includegraphics[width=\linewidth]{figures/example_from_tutorial.png}
    \caption{Example screenshot from the tutorial}
    \label{fig:ex_fig_tutorial}
\end{figure}


\subsection{Game Agents}

In this section, we describe the design and behavior of the various agents within the game. These agents, which include the player character, common enemies, and bosses, are crucial for creating a dynamic and engaging player experience. We will detail their underlying AI, movement patterns, and combat mechanics.

\subsubsection{Player Agent}

The player controls a knight character, navigating the dungeon and engaging in combat. The agent's mechanics (implemented following the tutorial mentioned in Section~\ref{sec:tutorial}) are designed to be intuitive and offer a variety of tactical options. Figure~\ref{fig:player_agent} shows the sprite for the hero.

\begin{figure}[H]
    \centering
    \includegraphics[height=0.2\linewidth]{figures/Hero_Idle.png}
    \caption{Sprite for the Player Agent.}
    \label{fig:player_agent}
\end{figure}

\paragraph{Movement and Abilities}
The player uses the standard WASD keys for top-down movement. Aiming is controlled by the mouse cursor, allowing for 360-degree targeting independent of movement direction. A key defensive ability is the dash, triggered by the space bar, which provides a short burst of invulnerability and speed. Dashing consumes stamina from a dedicated bar, which regenerates automatically over time, requiring players to manage its use strategically.

\paragraph{Combat and Weapons}
Attacks are initiated with a left-click. The player has access to three distinct weapons, which can be switched using the number keys (1-3):
\begin{itemize}
    \item \textbf{Sword (Key 1):} A melee weapon that performs a sweeping attack in a short arc in front of the player. It is ideal for close-quarters combat against multiple weak enemies. Deals 1 damage.
    \item \textbf{Bow (Key 2):} A long-range weapon that shoots a single arrow. The arrow is destroyed upon hitting an enemy, making it suited for picking off targets from a distance. Deals 1 damage.
    \item \textbf{Magic Staff (Key 3):} A medium-range weapon that fires a piercing beam of magic. The beam slowly travels through all enemies in its path, making it effective for dealing with lined-up groups. Deals 3 damage.
\end{itemize}
All successful attacks apply a knockback effect to enemies, providing a brief moment of crowd control and repositioning opportunity.

\paragraph{Health and Resources}
The player starts with 3 health points by default, represented by a health bar that decreases upon taking damage from enemy attacks. Health can be replenished by collecting hearts, which have a chance to drop from defeated enemies. Additionally, fallen enemies drop coins that can be collected.

\subsubsection{Basic Monsters Agents}

The behavior of common enemies is governed by a Finite State Machine (FSM) that dictates their actions based on the player's proximity. However, the main focus of our AI work in this project is on procedural content generation (PCG), generative AI, and reinforcement learning (RL) agents, which are introduced and covered in detail in later sections. As such, the FSM governing the basic monsters is purely inherited from the base game and intentionally kept very simple.

Each enemy agent operates in one of two states: \texttt{Roaming} or \texttt{Attacking}. When in the \texttt{Roaming} state, the enemy moves to randomly generated points within the dungeon. If the player enters a predefined \texttt{attackRange}, the enemy transitions to the \texttt{Attacking} state. While attacking, each enemy is subject to an \texttt{attackCooldown} timer, which prevents it from attacking continuously and enforces a minimum delay between consecutive attacks. The enemy will remain in the \texttt{Attacking} state until the player moves out of range, at which point it returns to \texttt{Roaming}. Figure~\ref{fig:enemy_fsm} illustrates the finite state machine (FSM) governing enemy behavior. The diagram shows the two main states (\texttt{Roaming} and \texttt{Attacking}) and the transitions triggered by the player's proximity.

\begin{figure}[H]
  \centering
  \includegraphics[width=0.6\linewidth]{figures/EnemyAI FSM.png}
  \caption{Finite State Machine (FSM) for enemy agents. Enemies switch between \texttt{Roaming} and \texttt{Attacking} states based on the player's distance.}
  \label{fig:enemy_fsm}
\end{figure}

Currently, there are two types of enemies implemented, as shown in Figure~\ref{fig:enemy_sprites}:

\begin{itemize}
    \item \textbf{Slimes:} These enemies represent the simplest form of the FSM. They do not have an explicit attack action. Instead, their ``attack'' is passive; they damage the player upon contact. Their \texttt{attackRange} is very small, meaning they only transition to the \texttt{Attacking} state when the player is nearly touching them, at which point contact damage is applied.
    \item \textbf{Ghosts:} These enemies are shooters that engage the player from a distance. When they enter the \texttt{Attacking} state, they trigger a shooting behavior. They can be configured to fire a burst of projectiles in a wide arc or in an oscillating, wave-like pattern. This allows for varied combat encounters depending on the specific instance of the Ghost enemy.
\end{itemize}

\begin{figure}[H]
    \centering
    \includegraphics[height=0.1\linewidth]{figures/Slime_Sheet.png}
    \includegraphics[height=0.1\linewidth]{figures/Ghost.png}
    \caption{Sprites for the Slime (left) and Ghost (right) enemies.}
    \label{fig:enemy_sprites}
\end{figure}

\subsubsection{Boss Agents}\label{sec:boss_agents}

The three main bosses in the game are designed to be significant challenges that test the player's skill. Their behavior is controlled by agents trained using Reinforcement Learning (RL), allowing them to react to the player's actions and create dynamic, unscripted encounters.

Our approach is to model the boss agents as enhanced, hostile versions of the player character. At their core, these bosses possess a similar set of abilities as the player's knight, but they are controlled by a trained RL agent.

Unlike the player, boss agents only have access to the Sword weapon; however, to compensate for this limitation, we increased their health points (7 HP instead of the player's 3), and their movement speed (from 4 to 5). To make them more menacing and visually distinct, we gave them a different color palette (see Figure~\ref{fig:boss_agent}).

\begin{figure}[H]
    \centering
    \includegraphics[height=0.2\linewidth]{figures/Villain_Idle.png}
    \caption{Sprite for the Boss Agents.}
    \label{fig:boss_agent}
\end{figure}

The technical specifics of the training process, reward functions, and agent architecture for these bosses are detailed further in Section~\ref{sec:rl_proposal}.

\section{Related Work}

\paragraph{PCG}
Game design increasingly takes advantage of procedural content and AI-driven systems. 
Several roguelike-inspired action games -- notably \emph{The Binding of Isaac}, \emph{Spelunky}, \emph{Dead Cells} and \emph{Hades} -- use procedural algorithms such as binary space partitioning or cellular automata to assemble dungeons and item layouts at runtime, to have high replayability. Large open-world games like \emph{Minecraft} and \emph{No Man's Sky} show how flexible PCG is~\cite{shaker2016procedural}.

\paragraph{Narrative generation}
\emph{Middle-earth: Shadow of Mordor/Shadow of War} introduced the ``Nemesis'' system, in which orc captains are procedurally generated and remember past encounters; they can return with scars or new ranks, creating unique personal stories.
Earlier AI-driven titles such as \emph{Façade} and \emph{Versu} started dynamic dialogue and multiple endings, while Ubisoft's \emph{Watch Dogs: Legion} showed how to write a dynamic script: the narrative team wrote twenty variations of every line and tied each version to a procedurally generated persona (e.g., a policeman says ``much obliged'', a young activist says ``appreciate it, fam''). 
Simulation-heavy games like \emph{Dwarf Fortress} and \emph{RimWorld} are described as story generators because their complex systems of characters, needs and events produce emergent stories rather than fixed plots~\cite{liapis2013sentient,buongiorno2024pangea}.

\paragraph{Generative music}
Dynamic soundtracks have been explored in both indie and AAA games.
In \emph{No Man's Sky}, audio director Paul Weir and the band 65daysofstatic created hundreds of musical fragments tagged by key, tempo and mood; the game's ``Pulse'' tool combines these elements in real time to generate soundscapes that react to what the player is doing.
A decade earlier, Brian Eno composed a fully procedural score for \emph{Spore}, making the game's music as generative as its evolving life forms~\cite{castellon2023generative}.

\paragraph{Reinforcement learning agents}
While many shipped games focus on hand-authored AI behaviors or simple difficulty scaling, there is a growing body of work exploring reinforcement learning (RL) agents trained for complex control tasks. AlphaStar, for example, used large-scale self-play to learn high-level strategies and unit micro-management in \emph{StarCraft II}, producing agents that could compete with professional players~\cite{vinyals2019alphastar}. Similarly, the Visual Doom AI Competitions challenged participants to train agents directly from raw pixels in \emph{Doom}, where the strongest entries relied on deep RL methods to achieve competitive performance~\cite{Wydmuch_2019}.


\section{Proposal}

The design of our game relies on a combination of procedural content generation (PCG) and adaptive AI systems to ensure that every run feels unique yet narratively coherent. 
The following subsections describe the main AI components that create the dungeon structure, generate narrative, generate adaptive music, and control enemy and boss behaviors.

\subsection{Procedural Dungeon Generation}
\label{subsec:procedural_dungeon}

\subsubsection{Dungeon Run Manager, Connectivity Modes, and BSP Generation}
\label{subsubsec:bsp_layout}

We first describe the \texttt{DungeonRunManager}, which orchestrates chapter progression and regeneration. We then describe how each chapter dungeon is generated using BSP and (internally) connected via corridors or (optionally) portals. Full pseudocode is given in Appendix~\ref{app:pcg_pseudocode}.

\paragraph{DungeonRunManager (chapter seeds, persistence, and transitions)}
The run is split into three chapters. At the start of the run, \texttt{DungeonRunManager} samples and stores a \texttt{layoutSeed} for each chapter. When loading a chapter, it applies the corresponding theme (see \ref{subsubsec:ca_rooms} and calls \\ \texttt{GenerateWithSeedAndPlacePlayer(seed)}. The manager also enforces persistence: defeating a boss marks the chapter as completed and updates a checkpoint index. On player death, if the player is past the last completed chapter, only the current chapter seed is resampled and regenerated; completed chapters remain unchanged.

\paragraph{Connectivity: corridors vs.\ portals}
We treat connectivity as a first-class choice because it affects both gameplay readability and our GenAI pipeline (Section \ref{subsec:genai_pipeline}). In \emph{corridor mode}, the generator physically carves paths between rooms, producing classic dungeon traversal. In \emph{portal mode}, connected rooms are linked by a portal transition, optionally reducing long camera confiner corridors and making room boundaries cleaner. We considered portal connectivity because: (i) it matches our asset set (portals already exist and look better than very long corridors in a small tileset),(ii) it makes it easier to capture consistent screenshots for GenAI (clean room views with fewer transitional hallway artefacts).

\begin{figure}[H]
  \centering
  \begin{minipage}[t]{0.48\linewidth}
    \centering
    \includegraphics[width=\linewidth]{figures/corridor_example.png}
    \caption{Corridor connectivity between rooms.}
    \label{fig:corridor_example}
  \end{minipage}
  \hfill
  \begin{minipage}[t]{0.48\linewidth}
    \centering
    \includegraphics[width=\linewidth]{figures/portal_example.png}
    \caption{Portal connectivity/exit portal example.}
    \label{fig:portal_example}
  \end{minipage}
\end{figure}

\paragraph{Boss-defeat exit portal}
Chapter progression is gated by an exit portal spawned only after the boss is defeated. Concretely, \texttt{OnBossDefeated()} marks the chapter as completed, shows a short victory UI, and calls \\ \texttt{SpawnChapterExitPortalAtBossRoom()}.This prevents skipping bosses, makes success visible to the player, and keeps transitions deterministic (use portal $\rightarrow$ \texttt{LoadChapter(current+1)}).

Figure~\ref{fig:run_transitions} shows the transition states controlled by \texttt{DungeonRunManager}.

\begin{figure}[H]
  \centering
  \begin{minipage}[t]{0.32\linewidth}
    \centering
    \includegraphics[width=\linewidth]{figures/death_screen.png}\\
    \small\textbf{(a)} Death screen (on player death).
  \end{minipage}
  \hfill
  \begin{minipage}[t]{0.32\linewidth}
    \centering
    \includegraphics[width=\linewidth]{figures/final_victory.png}\\
    \small\textbf{(b)} Final victory screen (after last boss).
  \end{minipage}
  \hfill
  \begin{minipage}[t]{0.32\linewidth}
    \centering
    \includegraphics[width=\linewidth]{figures/boss_portal.png}\\
    \small\textbf{(c)} Exit portal spawned after boss defeat.
  \end{minipage}

  \caption{Run transitions managed by \texttt{DungeonRunManager}: (a) death resets the current unsealed chapter, (b) final victory ends the run, and (c) boss defeat spawns the chapter exit portal.}
  \label{fig:run_transitions}
\end{figure}


\paragraph{BSP map layout generation}
Each chapter dungeon is generated by \\ \texttt{BSPMSTDungeonGenerator}. The generator operates on a grid of size \\ \texttt{mapWidth}$\times$\texttt{mapHeight}, recursively splits the space into rectangular leaves, and samples one room per terminal leaf when feasible. Splitting is adaptive to avoid degenerate leaves: letting $w=\texttt{bounds.width}$ and $h=\texttt{bounds.height}$, we bias the split orientation based on the aspect ratio ($w/h$ vs.\ $h/w$). Splits are only accepted if both children remain large enough (\texttt{minLeafSize}), so that we can later carve rooms of at least \texttt{minRoomSize} with margin.

\paragraph{Room carving}
For each terminal leaf, we compute candidate maxima
$\texttt{maxW}=\texttt{bounds.width}-2$ and $\texttt{maxH}=\texttt{bounds.height}-2$ and clamp them to $[\texttt{minRoomSize},\,\texttt{maxRoomSize}]$. If either is below \texttt{minRoomSize}, the leaf is skipped. Otherwise, room dimensions are sampled uniformly: \\
$\texttt{w}\sim\{\texttt{minRoomSize},\dots,\texttt{maxW}\}$ and
$\texttt{h}\sim\{\texttt{minRoomSize},\dots,\texttt{maxH}\}$,
and the bottom-left corner is sampled uniformly among valid placements so the room fits inside the leaf with a one-tile margin. Accepted rooms are painted to the tilemap and their floor cells are stored for subsequent wall placement and camera bounds.

\paragraph{Corridor graph (MST) and connectivity}
When using corridor mode, we connect rooms via a graph over room centres. For rooms $i,j$ with centres $\mathbf{c}_i,\mathbf{c}_j$, we assign edge weight
\[
w_{ij}=\lVert \mathbf{c}_i-\mathbf{c}_j\rVert_2.
\]
We compute a Minimum Spanning Tree using Kruskal's algorithm (Union-Find), guaranteeing that all rooms are reachable while minimizing total corridor length. Optionally, we add a small number of extra edges (short non-MST connections) to create loops and reduce linearity.

\paragraph{Walls, outside tiles, and camera bounds}
Finally, we place walls by inspecting neighbours around floor cells, fill a border of ``outside'' tiles to avoid empty camera regions, and derive camera confiner bounds from the generated floor region.



% Cellular Automata Room Population Section

\subsubsection{Cellular Automata Room Population}
\label{subsubsec:ca_rooms}

While the BSP algorithm described in Section~\ref{subsubsec:bsp_layout} generates the spatial structure of the dungeon, the rooms themselves need to be populated with different environmental features, obstacles, enemies, and decorations to create fun gameplay spaces. This section describes the cellular automata-based system we used to procedurally populate rooms with features that look natural and that vary across the game's three chapters.

\paragraph{Design Motivation}

Procedural content generation risks creating environments that feel repetitive or artificial if objects are placed uniformly or on regular grids. To address this, we employ \textbf{Cellular Automata} (CA). This is a model where cells evolve based on local neighborhood rules. CA is widely used to generate organic clustering patterns for many different room features. It naturally produces irregular distributions that avoid the monotonic random or grid-based placement.

We apply CA to generate three distinct room features: lakes, portal-seeded paths to guide player navigation, and clusters of small bushes for visual variety.

For larger obstacles such as trees and buildings, we chose to use randomized placement with collision detection rather than CA, as we didn't want to put so many of them that they would dominate the room or block navigation.

\paragraph{Chapter Theme System}

In sync with the game's three-chapter structure described in Section~\ref{sec:scenario}, we designed a \textbf{Chapter Theme} system that encapsulates all visual and gameplay assets for each chapter. Each theme is implemented as a Unity ScriptableObject containing:

\begin{itemize}
    \item \textbf{Visual Assets}: Floor tiles, water tiles and path tiles
    \item \textbf{Prefabs}: Blocking obstacles (trees, bushes, buildings), non-blocking decorations (grass, flowers, small bushes), and enemy types
    \item \textbf{CA Parameters}: Configurable density, iteration count, and the survival/birth thresholds for each CA application
    \item \textbf{Atmosphere Settings}: Ambient lighting color and intensity
\end{itemize}

This architecture allows each chapter to have a distinct visual identity while reusing the same generation algorithms. We implemented the following three themes: Chapter 1 (Forest), Chapter 2 (Night), and Chapter 3 (Hell).

By changing the theme, the entire visual atmosphere of the dungeon transforms while maintaining consistent generation logic.

\begin{figure}[H]
    \centering
    \begin{minipage}[t]{0.32\linewidth}
        \centering
        \includegraphics[width=\linewidth,height=4cm,keepaspectratio]{figures/ca_screenshots/forest_theme.png}
        \caption{Forest Theme}
    \end{minipage}
    \hfill
    \begin{minipage}[t]{0.32\linewidth}
        \centering
        \includegraphics[width=\linewidth,height=4cm,keepaspectratio]{figures/ca_screenshots/night_theme.png}
        \caption{Night Theme}
    \end{minipage}
    \hfill
    \begin{minipage}[t]{0.32\linewidth}
        \centering
        \includegraphics[width=\linewidth,height=4cm,keepaspectratio]{figures/ca_screenshots/hell_theme.png}
        \caption{Hell Theme}
    \end{minipage}
    \caption{The three chapter themes. Same CA algorithms, different visual assets and atmosphere.}
    \label{fig:chapter_themes}
\end{figure}

\paragraph{Cellular Automata Algorithm}

In the core CA algorithm we followed a standard approach: each cell in a 2D grid evolves over multiple iterations based on the state of its neighbors. The algorithm takes as input the room's dimensions, an initial density parameter, and CA rules (survival and birth thresholds). See ~\ref{app:ca_pseudocode} for the complete pseudocode.

The algorithm uses a \textbf{Moore neighborhood} (the 8 cells surrounding each cell) to count neighbors. Cells with at least \texttt{surviveMin} neighbors remain alive, and dead cells with at least \texttt{birthMin} neighbors become alive. This produces natural clustering: isolated cells die off, while groups of cells stabilize into blob-like shapes.

In the final cleanup step we remove ``orphan'' tiles, these are single tiles or very small clusters that would cause strange borders with the RuleTile rendering system. This is done by iteratively removing any tile with fewer than 3 neighbors, repeating for up to 2 passes.

\paragraph{Application 1: Lakes}

Some rooms are designated as environmental hazard rooms and contain lakes generated with CA. We first place floor tiles everywhere, then generate the lake shape using CA with \texttt{density=0.4}, \texttt{iterations=6}, \texttt{surviveMin=5}, and \texttt{birthMin=4}.

The high \texttt{surviveMin=5} threshold helps create larger bodies of water instead of many small puddles. We then use Unity's RuleTile system to place the water tiles, which automatically renders the appropriate border tiles based on neighboring tiles.

Water cells are marked as occupied so that enemies and obstacles don't spawn in the water. Instead, enemies are placed on the shore around the lake.

\begin{figure}[H]
    \centering
    \includegraphics[width=0.35\linewidth]{figures/ca_screenshots/lake_example.png}
    \caption{Example of CA-generated lake with automatic RuleTile borders. The high survival threshold creates cohesive water bodies.}
    \label{fig:ca_lake}
\end{figure}

\paragraph{Application 2: Paths}

To help guide players between rooms, we generate dirt paths using CA with a novel \textbf{portal-seeded initialization} where cells near portals (within radius $r=3$ tiles) have high initial density (\texttt{portalPathDensity=0.9}), while cells away from portals have low ambient density (\texttt{ambientPathDensity=0.2}).

This creates natural-looking paths that emerge from room entrances and gradually fade into the environment. The CA smoothing (\texttt{iterations=4}, \texttt{surviveMin=3}, \texttt{birthMin=2}) connects high-density portal regions into continuous paths while allowing some branching and irregularity.

Paths use the chapter's \texttt{pathRuleTile}, which renders dirt paths with grass borders. Paths are non-blocking (enemies and players can walk over them), serving purely as visual navigation cues.

\begin{figure}[H]
    \centering
    \includegraphics[width=0.35\linewidth]{figures/ca_screenshots/paths_example.png}
    \caption{Portal-seeded paths emerging from room entrances. High density near portals creates natural-looking navigation guides.}
    \label{fig:ca_paths}
\end{figure}

\paragraph{Application 3: Decorations}

Small bushes are placed using CA to create natural clustering. The decoration CA uses moderate density (\texttt{decorationDensity=0.3}) and runs for 5 iterations with \texttt{surviveMin=4} and \texttt{birthMin=3}.

To prevent decorations from spawning too close to walls (where they would look cause overlaps with the border of the rooms), the initial grid excludes a 2-tile border around the room perimeter. Additionally, decorations avoid positions already occupied by obstacles, enemies, or water.

\paragraph{Random Placement for Large Obstacles}

Larger obstacles such as trees, rocks, and buildings are placed using randomized position sampling rather than CA. These objects have complex collision shapes (some buildings span multiple tiles) and require explicit spacing to prevent overlaps.

The placement algorithm:
\begin{enumerate}
    \item Sample a random position within the room (with a margin from walls)
    \item Calculate the obstacle's footprint: all grid cells it will occupy based on its collider bounds, plus a 0.5-tile padding for visual spacing
    \item Check if any footprint cells are already occupied
    \item If placement is valid, instantiate the obstacle and mark all footprint cells as occupied
    \item Repeat until the target number of obstacles is placed or maximum attempts reached
\end{enumerate}

\paragraph{Room Population Pipeline}

The complete room population process follows a carefully ordered pipeline to ensure features do not conflict:

\begin{enumerate}
    \item \textbf{Floor tiles}: Base layer placed everywhere
    \item \textbf{Lakes} (hazard rooms only): CA-generated water with borders
    \item \textbf{Paths}: Portal-seeded CA paths placed on a separate tilemap layer
    \item \textbf{Blocking obstacles}: Trees, rocks, buildings placed with collision detection
    \item \textbf{Enemies}: Placed avoiding obstacles and water
    \item \textbf{Decorations}: CA-generated small bushes and grass filling remaining space
\end{enumerate}

Each step marks occupied positions in a shared \texttt{HashSet} to ensure subsequent steps avoid placing objects on top of each other.


% Include the GenAI pipeline section from external file
% Generative AI Pipeline Section for Sprint 3 Final Report

\subsection{Generative AI Pipeline}
\label{subsec:genai_pipeline}

The generative AI pipeline creates dynamic narrative, music, and voice content for each chapter of the game. This section describes the final implementation and the design evolution from our initial Sprint 2 prototype to the production-ready system.

\subsubsection{Design Evolution: From Runtime to Pre-Generation}
\label{subsubsec:design_evolution}

\paragraph{Original Vision (Sprint 2)}

Our initial design in Sprint 2 was ambitious: a fully dynamic, vision-driven content generation system. The pipeline would:

\begin{enumerate}
    \item Capture a screenshot of each procedurally generated room
    \item Analyze the screenshot using a vision-language model (moondream2) to extract semantic features (environment type, atmosphere, visible objects)
    \item Feed this visual analysis to an LLM (llama2) to generate contextual narrative content
    \item Generate ambient music matching the analyzed mood
    \item Deliver all content to Unity via HTTP requests in real-time
\end{enumerate}

This approach was elegant in theory---content would dynamically adapt to the visual characteristics of each unique room layout. We implemented a FastAPI server with dedicated endpoints for vision analysis, narrative generation, and music synthesis.

\paragraph{Challenges Encountered}

Testing revealed a fundamental issue with the runtime HTTP server approach: \textbf{development iteration time}. Every code change required restarting the server, reloading all AI models into GPU memory, and running the full generation pipeline. With model loading taking 30--60 seconds and generation taking several minutes per test, the feedback loop became prohibitively slow. Simple prompt adjustments or bug fixes that should take minutes instead consumed hours of waiting.

Additionally, we encountered secondary issues:

\begin{itemize}
    \item \textbf{Vision Model Limitations}: The moondream2 vision model struggled with our pixel art aesthetic, often misidentifying dungeon elements or producing generic descriptions that didn't meaningfully differentiate between rooms.

    \item \textbf{Room-Level Granularity Mismatch}: Generating unique content for each room (potentially 5--10 per run) created inconsistency. NPCs in adjacent rooms might have contradictory dialogue, and the overall narrative felt fragmented.
\end{itemize}

\paragraph{The Pivot: Chapter-Based Pre-Generation}

We fundamentally redesigned the system based on two key insights:

\begin{enumerate}
    \item \textbf{Content should align with game structure}: Our game has three chapters, each with a distinct theme and boss. Generating content per-chapter rather than per-room creates natural narrative coherence and matches the checkpoint system.

    \item \textbf{Generation can happen offline}: Since chapter themes are defined in advance (forest $\rightarrow$ night $\rightarrow$ fire), we can generate all content before the game runs, eliminating runtime latency entirely.
\end{enumerate}

This led to the \textbf{pre-generation architecture}: a Python script (\texttt{generate.py}) runs once during development, produces all assets, and saves them to Unity's \texttt{StreamingAssets} folder. The shipped game simply loads these pre-generated files.

\paragraph{What We Removed and Why}

\begin{itemize}
    \item \textbf{Vision Analysis}: No longer needed. Our current set of visual assets is limited and not rich enough to benefit from a vision model analyzing the generated levels---chapter themes provide sufficient context. The vision model added complexity without meaningfully improving output quality.

    \item \textbf{HTTP Server}: Replaced by a simple generation script. The server approach made development iteration painfully slow---every change required a full restart and model reload cycle.

    \item \textbf{Room-by-room Generation}: Replaced by chapter-based content. Three coherent chapter narratives are more impactful than potentially dozens of disconnected room descriptions.
\end{itemize}

\paragraph{What We Added}

\begin{itemize}
    \item \textbf{Voice Generation}: With content generated offline, we could add text-to-speech synthesis without worrying about runtime latency. A narrator voice significantly enhances immersion.

    \item \textbf{Chapter-Themed Music}: Instead of trying to match music to visual features, we designed specific music prompts for each chapter's atmosphere (calm forest, tense night, dramatic finale).

    \item \textbf{Quality Control}: Pre-generation allows us to review and regenerate content that doesn't meet quality standards before shipping.
\end{itemize}

\subsubsection{Final Architecture}
\label{subsubsec:genai_architecture}

The final pipeline consists of three generation stages executed sequentially:

\begin{enumerate}
    \item \textbf{Narrative Generation}: An LLM generates dialogue lines for each chapter based on the story context and chapter theme.
    \item \textbf{Music Generation}: A text-to-music model creates ambient background music matching each chapter's atmosphere.
    \item \textbf{Voice Generation}: A text-to-speech model converts the generated dialogue into narrated voice clips.
\end{enumerate}

Figure~\ref{fig:genai_pipeline} illustrates the complete pre-generation and runtime loading architecture.

\begin{figure}[H]
\centering
\begin{tikzpicture}[
    node distance=1cm and 1.8cm,
    box/.style={rectangle, draw, rounded corners, minimum width=2.2cm, minimum height=0.7cm, align=center, font=\footnotesize},
    model/.style={rectangle, draw, fill=blue!15, rounded corners, minimum width=2cm, minimum height=0.7cm, align=center, font=\footnotesize},
    file/.style={rectangle, draw, fill=green!15, minimum width=1.8cm, minimum height=0.6cm, align=center, font=\scriptsize},
    arrow/.style={->, >=stealth, thick},
    section/.style={rectangle, draw, rounded corners, fill=gray!10, inner sep=0.4cm}
]

% Pre-generation section title
\node[font=\small\bfseries] (pregentitle) {Pre-Generation (Python)};

% Config input
\node[box, below=0.6cm of pregentitle] (config) {config.json};

% Models column
\node[model, right=1.2cm of config] (ollama) {Ollama\\mistral-nemo};
\node[model, below=0.6cm of ollama] (musicgen) {MusicGen};
\node[model, below=0.6cm of musicgen] (tts) {XTTS v2};

% Output files column
\node[file, right=1.2cm of ollama] (narrative) {narrative.json};
\node[file, right=1.2cm of musicgen] (music) {music.wav};
\node[file, right=1.2cm of tts] (voice) {voice\_*.wav};

% Arrows for pre-generation
\draw[arrow] (config) -- (ollama);
\draw[arrow] (config.south) -- ++(0,-0.3) -| (musicgen.west);
\draw[arrow] (config.south) -- ++(0,-0.3) -| (tts.west);
\draw[arrow] (ollama) -- (narrative);
\draw[arrow] (musicgen) -- (music);
\draw[arrow] (tts) -- (voice);

% StreamingAssets box around output files
\node[rectangle, draw, dashed, fit=(narrative)(music)(voice), inner sep=0.25cm, label={[font=\scriptsize]below:StreamingAssets/}] (assets) {};

% Separator line
\draw[thick, gray, dashed] ([yshift=-1.2cm]config.south west) -- ++(10.5cm,0);

% Runtime section title - more space below pre-generation
\node[font=\small\bfseries, below=2.8cm of config.south west, anchor=west, xshift=-0.5cm] (runtimetitle) {Runtime (Unity C\#)};

% Runtime components
\node[box, right=1.5cm of runtimetitle] (loader) {ContentLoader};
\node[box, right=1.2cm of loader] (managers) {Managers};
\node[box, right=1.2cm of managers] (ui) {DialogueUI};

% Arrows for runtime
\draw[arrow, dashed] (assets.south) -- ++(0,-0.6) -| (loader.north);
\draw[arrow] (loader) -- (managers);
\draw[arrow] (managers) -- (ui);

\end{tikzpicture}
\caption{Generative AI pipeline architecture. The pre-generation phase runs offline using Python and GPU-accelerated models. Generated assets are saved to StreamingAssets and loaded at runtime by Unity.}
\label{fig:genai_pipeline}
\end{figure}

\paragraph{Model Selection Rationale}

Table~\ref{tab:ai_models_final} summarizes the AI models used in the final implementation. Each model was selected based on specific criteria:

\begin{table}[H]
\centering
\caption{AI models used in the generative pipeline}
\label{tab:ai_models_final}
\begin{tabular}{lccp{5cm}}
\hline
\textbf{Component} & \textbf{Model} & \textbf{Parameters} & \textbf{Selection Rationale} \\
\hline
Narrative & mistral-nemo & 12B & Fast inference, excellent instruction following for structured JSON output \\
Music & MusicGen Small~\cite{copet2023musicgen} & 300M & Best quality-to-speed ratio for ambient game music \\
Voice & XTTS v2~\cite{casanova2024xtts} & 467M & Multilingual voice cloning with natural prosody \\
\hline
\end{tabular}
\end{table}

We chose \textbf{mistral-nemo}~\cite{mistralnemo} over llama2 (used in Sprint 2) because llama2 struggled significantly with instruction following. Despite explicit prompts requesting structured JSON output, llama2 frequently produced malformed responses: missing required fields, adding unsolicited commentary, or ignoring the specified format entirely. This resulted in retry rates of approximately 40\%, wasting generation time and requiring complex parsing logic to handle edge cases. Mistral-nemo, in contrast, follows structured output instructions reliably, reducing retry rates to under 10\%. The model runs locally via Ollama~\cite{ollama}, eliminating API costs and ensuring data privacy.

\textbf{MusicGen Small} was selected after testing multiple configurations. While the Medium (1.5B) variant produces slightly higher quality output, the Small variant (300M parameters) generates acceptable quality at 4x the speed, which is important when generating music for multiple chapters. The guidance scale of 3.5 balances prompt adherence with musical coherence.

\textbf{XTTS v2} was chosen for voice generation because it produces reasonably natural multilingual speech from a single reference audio sample. The model supports voice cloning with as little as 6 seconds of reference audio, making it practical for our use case. However, XTTS v2 is not without limitations: the generated speech occasionally exhibits unnatural pacing, and the overall voice quality remains noticeably synthetic compared to professional recordings.

\paragraph{Model Selection Challenges}

Selecting appropriate generative models required extensive experimentation, as the quality gap between state-of-the-art cloud APIs and locally-runnable open-source models remains significant.

For text-to-speech, we initially tested \textbf{Parler-TTS}, which promised controllable voice generation through natural language descriptions. However, the model produced noticeable audio artifacts---particularly unwanted noises at the end of sentences and occasional ``robotic'' prosody that broke immersion. The voices often sounded unnatural, with inconsistent pacing and awkward pauses that would be distracting during gameplay. After testing multiple Parler-TTS configurations, we switched to XTTS v2, which produces more natural-sounding speech through voice cloning rather than description-based generation.

For music generation, we experimented with MusicGen's larger variants (Medium at 1.5B and Large at 3.3B parameters) but found that the quality improvement did not justify the 4--8x increase in generation time. The Small variant produces audio that is sufficiently atmospheric for background game music, where players focus primarily on gameplay rather than musical nuance.

For narrative generation, commercial APIs (OpenAI, Anthropic) produce superior text quality but introduce ongoing costs, latency dependencies, and privacy concerns. Running models locally via Ollama eliminates these issues at the cost of slightly lower output quality, which we mitigated through careful prompt engineering and retry logic.

\subsubsection{Chapter-Based Content Structure}
\label{subsubsec:chapter_content}

Content is organized by chapter rather than by room. Each of the three chapters has its own theme, narrative arc, and atmospheric music. This structure aligns with the game's checkpoint system described in Section~\ref{sec:scenario}: defeating a boss freezes that chapter's content.

The \texttt{config.json} file defines each chapter's generation parameters:

\begin{lstlisting}[language=json, basicstyle=\ttfamily\scriptsize, frame=single]
{
  "chapters": [
    {
      "name": "The Verdant Prison",
      "setting": "A sunlit forest that hides a dark truth",
      "boss": "Thornback",
      "environment": "sunlit forest with ancient trees and ruins",
      "story": "The first layer of your prison appears peaceful...",
      "music_prompt": "calm fantasy ambient, gentle flute and strings"
    },
    ...
  ]
}
\end{lstlisting}

The generation script produces the following output structure:
\begin{lstlisting}[basicstyle=\ttfamily\scriptsize, frame=single]
StreamingAssets/GeneratedContent/
  manifest.json           # Content manifest with seed and file paths
  chapters/
    chapter_0/
      narrative.json      # NPC name and dialogue lines
      music.wav           # 30-second loopable ambient track
      voice_0.wav         # Narrated first dialogue line
      voice_1.wav         # Narrated second dialogue line
      voice_2.wav         # Narrated third dialogue line
    chapter_1/
      ...
    chapter_2/
      ...
\end{lstlisting}

\subsubsection{Narrative Generation}
\label{subsubsec:narrative_gen_final}

The narrative generator creates contextual dialogue for a narrator character who introduces each chapter. Unlike the room-by-room approach in Sprint 2, narrative is now chapter-scoped to match the game's three-act structure.

The LLM receives a prompt constructed from the chapter's theme and story context:

\begin{lstlisting}[basicstyle=\ttfamily\scriptsize, frame=single]
You are a fantasy game narrator with a deep, dramatic voice.
Setting: {chapter_name}. The boss is {boss}.
Context: {progress_context}

Write exactly 3 dramatic narrator lines (full sentences, 15-25 words each)
that set the mood and hint at the danger ahead. Be atmospheric and mysterious.
Output ONLY the 3 numbered lines:
1.
2.
3.
\end{lstlisting}

The generator parses the numbered output and validates that exactly three non-empty lines are produced. A retry mechanism with seed perturbation handles occasional malformed outputs.

Example generated narrative for Chapter 0 (``The Verdant Prison''):

\begin{lstlisting}[language=json, basicstyle=\ttfamily\scriptsize, frame=single]
{
  "roomIndex": 0,
  "environment": "sunlit forest with ancient trees and overgrown ruins",
  "npc": {
    "name": "The Narrator",
    "dialogue": [
      "The forest welcomes you with golden light, but darkness lurks beneath
       the ancient canopy.",
      "Thornback, the Guardian of Growth, has twisted nature itself into a
       weapon against all who enter.",
      "Only by defeating the guardian can you proceed deeper into the realm
       and find your way home."
    ]
  }
}
\end{lstlisting}

\subsubsection{Music Generation}
\label{subsubsec:music_gen_final}

Each chapter has distinct ambient music generated using MusicGen. The music prompts are designed to evoke the emotional tone of each chapter's environment:

\begin{itemize}
    \item \textbf{Chapter 0} (Forest): ``calm fantasy ambient music, gentle flute and strings, forest atmosphere, mysterious, 90 bpm''
    \item \textbf{Chapter 1} (Night): ``tense ambient music, low strings and soft piano, night atmosphere, suspenseful and eerie, 80 bpm''
    \item \textbf{Chapter 2} (Fire): ``dramatic orchestral music, building tension, brass and percussion, climactic and intense, 110 bpm''
\end{itemize}

All prompts include the suffix ``loopable, no vocals'' to ensure the generated music is suitable for continuous background playback. The \texttt{ChapterMusicManager} component handles crossfading between chapter tracks and volume ducking during dialogue.

\subsubsection{Voice Generation}
\label{subsubsec:voice_gen}

Voice generation converts the narrative dialogue lines into spoken audio, creating an immersive narrator experience. We use XTTS v2~\cite{casanova2024xtts}, a multilingual text-to-speech model that can clone a voice from a single reference sample.

\paragraph{Why Voice Generation?}

Adding voiced narration significantly enhances immersion compared to text-only dialogue. The narrator's voice reinforces the game's atmospheric tone and provides consistent delivery across all chapters. While professional voice acting would be ideal, AI-generated voice provides a practical solution for procedurally generated content where recording every possible line is infeasible.

\paragraph{Reference Voice Selection}

We use a single 15-second reference audio sample of documentary-style narration to establish the narrator's voice characteristics. For testing purposes, we extracted a voice sample from a BBC wildlife documentary\footnote{\url{https://www.youtube.com/watch?v=ErrcHBlxOBs}}. XTTS v2 clones the prosody, pitch, and speaking style from this reference while generating the actual game dialogue. This ensures consistency across all generated voice lines while matching the dramatic, atmospheric tone we wanted for the narrator.

\paragraph{Ethical Considerations: Voice Ownership}

The ease with which modern TTS models can clone voices raises significant ethical and legal questions. With only a few seconds of audio, it is now trivial to replicate someone's voice without their consent. In our case, we used a BBC narrator's voice for development and testing, but this practice highlights broader concerns about voice property and ownership in the age of generative AI. Professional voice actors face potential displacement as their distinctive voices can be copied and used indefinitely without compensation. For a production release, original voice recordings with proper licensing would be essential to avoid intellectual property issues.

\paragraph{Generation Process}

For each dialogue line, the voice generator:
\begin{enumerate}
    \item Loads the reference audio sample
    \item Synthesizes speech for the dialogue text
    \item Saves the output as a WAV file
\end{enumerate}

Voice files are named by chapter and line index (\texttt{voice\_0.wav}, \texttt{voice\_1.wav}, etc.) and loaded by the \texttt{DialogueUI} component during gameplay.

\subsubsection{Runtime Integration}
\label{subsubsec:runtime_integration}

The Unity side of the pipeline consists of several components that load and play the pre-generated content:

\paragraph{GeneratedContentLoader}
This singleton loads the content manifest and provides asynchronous methods for loading narrative JSON, music WAV files, and voice clips. It uses Unity's \texttt{UnityWebRequest} API to load files from \texttt{StreamingAssets}, which works consistently across all platforms.

\paragraph{LLMNarrativeGenerator}
Despite its name (retained for compatibility), this component no longer calls an LLM at runtime. Instead, it loads pre-generated narratives from JSON files and caches them for retrieval by other systems.

\paragraph{ChapterMusicManager}
Manages background music playback with crossfading support. When the player enters a new chapter, this component fades out the current track and fades in the new chapter's music. During dialogue sequences, the music volume is reduced to ensure voice clarity.

\paragraph{DialogueUI}
Displays the narrator dialogue with synchronized voice playback. The component preloads voice clips for the current chapter to minimize playback latency. Voice clips are played sequentially as the player advances through dialogue lines, with an option to skip.

\paragraph{AIManagerBootstrap}
A utility component that automatically creates all AI-related managers at runtime using \texttt{RuntimeInitializeOnLoadMethod}. This ensures the managers exist before any scene-specific code attempts to access them, solving initialization order issues.

\subsubsection{Generation Performance}
\label{subsubsec:generation_performance}

Table~\ref{tab:generation_times} shows typical generation times on our development hardware (NVIDIA RTX 5070 Ti, 16GB VRAM):

\begin{table}[H]
\centering
\caption{Content generation times per chapter}
\label{tab:generation_times}
\begin{tabular}{lcc}
\hline
\textbf{Content Type} & \textbf{Time} & \textbf{Parallelizable} \\
\hline
Narrative (3 lines/chapter) & $\sim$10 seconds/chapter & Yes \\
Music (30 seconds audio) & 16--20 seconds/chapter & No (GPU-bound) \\
Voice (3 lines/chapter) & 8--15 seconds/chapter & No (GPU-bound) \\
Model loading (first run) & $\sim$50 seconds & -- \\
\hline
\textbf{Full game (3 chapters)} & \textbf{$\sim$2.5 minutes} & -- \\
\hline
\end{tabular}
\end{table}

Narrative generation is parallelized across chapters using Python's \texttt{ThreadPoolExecutor}, as the Ollama server can handle concurrent requests. Music and voice generation are executed sequentially since they compete for GPU memory.

\subsubsection{Removed Components}
\label{subsubsec:removed_components}

The following Sprint 2 components were removed in the final implementation:

\begin{itemize}
    \item \textbf{Vision Analysis (moondream2)}: Originally used to analyze room screenshots and generate semantic descriptions. Removed because our limited asset set does not provide enough visual variety to benefit from vision-based analysis---chapter themes suffice.
    \item \textbf{FastAPI HTTP Server}: Replaced by the pre-generation script. The server architecture made development iteration extremely slow, as every code change required restarting and reloading all models.
    \item \textbf{Room-by-room Generation}: Replaced by chapter-based generation to align with the game's three-act checkpoint structure.
\end{itemize}

These simplifications reduced system complexity while improving the player experience through faster load times and more consistent content quality.


\subsection{Reinforcement Learning for Boss Behavior}\label{sec:rl_proposal}

Boss behavior is trained through Reinforcement Learning (RL) to produce adaptive and challenging encounters. As mentioned in Section~\ref{sec:boss_agents}, these bosses are modeled as replicas of the player character, possessing similar movement and attack capabilities.

The agents controlling these bosses are designed to mimic the player's capabilities while exhibiting complex human-like strategies. The setup for their training takes place in a custom Arena environment where two instances of the agents can fight each other, illustrated in Figure~\ref{fig:rl_training_arena}.

Rather than employing pure self-play from the start, the agents are first pre-trained against a heuristic opponent to establish a solid foundation of basic combat mechanics. The heuristic behavior moves directly towards or away from the agent (determined stochastically by an aggressiveness parameter $\alpha = 0.75$) and constantly attempts to attack in the direction of its target. After this pre-training phase, the agents are pitted against each other in self-play to develop more sophisticated strategies and counter-strategies.

\begin{figure}[h]
    \centering
    \includegraphics[width=0.6\linewidth]{figures/rl_training_arena.png}
    \caption{Screenshot of the Arena Unity Scene used for RL training, where the boss agent learns through self-play against a copy of itself.}
    \label{fig:rl_training_arena}
\end{figure}

The RL agents are implemented using the Unity ML-Agents Toolkit, which provides a framework for training intelligent agents in Unity environments. The specific software architecture and implementation details are discussed in Section~\ref{sec:rl_methodology_and_results}.

\subsubsection{Observation Space}

The observation space consists of 18 continuous values that provide the agent with comprehensive information about the combat state All values are normalized to $[-1, 1]$ range:

\begin{enumerate}
    \item \textbf{Own position} (2 values): The agent's $(x, y)$ coordinates, normalized based on arena bounds.
    \item \textbf{Distance to enemy} (2 values): The relative position vector to the opponent.
    \item \textbf{Own velocity} (2 values): The agent's current velocity vector, normalized by maximum possible speed (during dash).
    \item \textbf{Enemy velocity} (2 values): The opponent's velocity vector, enabling prediction of enemy movement.
    \item \textbf{Own health} (1 value): Current health as a fraction of maximum health.
    \item \textbf{Enemy health} (1 value): Opponent's health fraction.
    \item \textbf{Own stamina} (1 value): Current stamina level (used for dashing).
    \item \textbf{Own weapon index} (1 value): Index of currently equipped weapon.
    \item \textbf{Enemy weapon index} (1 value): Opponent's equipped weapon index.
    \item \textbf{Own attack cooldown} (1 value): Remaining cooldown time..
    \item \textbf{Enemy attack cooldown} (1 value): Opponent's cooldown state, enabling strategic timing of attacks and dodges.
    \item \textbf{Own aim angle} (1 value): Current weapon aim direction.
    \item \textbf{Enemy aim angle} (1 value): Opponent's aim direction, useful for anticipating attacks.
    \item \textbf{Angle to enemy} (1 value): The angle from the agent to the opponent.
\end{enumerate}

\subsubsection{Action Space}

The action space combines discrete and continuous components:

\begin{itemize}
    \item \textbf{Discrete actions} (3 branches):
    \begin{itemize}
        \item \textit{Movement} (5 actions): Stand still, move up, down, left, or right.
        \item \textit{Attack} (2 actions): No action, or attack with sword.
        \item \textit{Dash} (2 actions): No dash, or perform dash.
    \end{itemize}
    \item \textbf{Continuous actions} (1 value):
    \begin{itemize}
        \item \textit{Aim angle}: A continuous value in $[-1, 1]$ that controls weapon aiming direction. The mapping is defined such that $0$ corresponds to aiming upward ($90^{\circ}$), with positive values rotating counterclockwise and negative values rotating clockwise. This symmetric representation around the vertical axis facilitates learning of balanced left-right behaviors.
    \end{itemize}
\end{itemize}

\subsubsection{Reward Function}

The reward function $R$ is designed to encourage aggressive yet strategic behavior. The final reward values used are:

\begin{itemize}
    \item \textbf{Victory reward}: $r_{\mathrm{win}} = +1.0$ for defeating the opponent.
    \item \textbf{Defeat penalty}: $r_{\mathrm{lose}} = -0.8$ for being defeated.
    \item \textbf{Hit reward}: $r_{\mathrm{hit}} = +0.5$ for successfully landing an attack.
    \item \textbf{Damage penalty}: $r_{\mathrm{damage}} = -0.4$ for receiving damage.
    \item \textbf{Proximity reward}: $r_{\mathrm{prox}} = +0.001 \cdot \cos(\theta_v)$ per timestep, where $\theta_v$ is the angle between the agent's velocity vector and the direction to the enemy. This formulation rewards movement \emph{towards} the opponent rather than simply being close, which prevents orbiting behaviors where agents circle each other without engaging.
    \item \textbf{Time penalty}: $r_{\mathrm{time}} = -0.001$ per timestep to encourage efficient combat resolution.
\end{itemize}


The cumulative reward for an episode is thus:
\begin{equation}
    R_{\mathrm{episode}} = r_{\mathrm{outcome}} + \sum_{t=1}^{T} \left( r_{\mathrm{hit}}^{(t)} + r_{\mathrm{damage}}^{(t)} + r_{\mathrm{prox}}^{(t)} + r_{\mathrm{time}} \right)
\end{equation}
where $r_{\mathrm{outcome}} \in \{r_{\mathrm{win}}, r_{\mathrm{lose}}\}$ and $T$ is the episode length.

\subsubsection{Training Algorithm}

For the training process, we use the Proximal Policy Optimization (PPO) algorithm~\cite{schulman2017proximal}, known for its stability and effectiveness in continuous control tasks. We use the default ML-Agents hyperparameters with the following modifications:
\begin{itemize}
    \item \textbf{Batch size}: 2048 (increased from default 1024 for more stable gradient estimates).
    \item \textbf{Buffer size}: 20480 (increased proportionally with batch size).
    \item \textbf{Self-play parameters}: We configure the self-play mechanism with a window of 5 past policies, swapping opponents every 10,000 steps, and maintaining a 50\% probability of playing against the latest model versus a historical snapshot.
\end{itemize}

The training methodology, including the pre-training strategy against heuristic opponents and the experimental results, are detailed in Section~\ref{sec:rl_methodology_and_results}.

\section{Conclusion}
In this project, we aimed to build a 2D RPG where as much as possible is \emph{non-scripted} across runs. 
To that end, we combined three AI-driven components that replace hand-authored content at different levels: procedural generation for level structure and gameplay spaces (PCG), generative models for audio and narrative content (GenAI), and reinforcement learning for boss behaviour (RL). 
The key design requirement was not just variety, but \emph{playable} variety: regeneration must preserve navigation readability and prevent content clashes.

Our main system-level contribution is a chapter-based run structure with boss-gated checkpoints. Chapters act as the unit of persistence: after a boss is defeated, everything the player has cleared so far (layout, encounters, and generated content for that chapter) remains fixed for the rest of the run, while failures only trigger regeneration of the current, unfinished chapter. This preserves a sense of forward progress and stable pacing, while still delivering high replayability through controlled, localized randomness.

On the PCG side, we generate chapter layouts and then populate rooms with a theme-conditioned pipeline. 
Cellular automata is used to create organic clustering for features such as lakes, paths, and decorations, avoiding the artificial look of uniform random placement, while larger obstacles are placed with collision-aware sampling to preserve traversability. 

On the GenAI side, we use generative models to replace manually authored narrative and audio while keeping outputs consistent with the current chapter theme. Rather than generating content per room (which is brittle under frequent regeneration), we generate chapter-scoped narrator lines, ambient music, and voice that remain stable within a chapter instance and therefore do not contradict what the player is seeing. This design makes the generated content feel intentional instead of noisy, and it avoids runtime latency or mid-run tone shifts that would hurt playability.

Finally, the Reinforcement Learning component demonstrated that boss agents can learn complex combat behaviors without explicit scripting. Our experiments showed that a two-phase training strategy (pre-training against a heuristic opponent followed by self-play refinement) was superior to pure self-play. This methodology prevented the agents from converging on passive strategies and produced bosses that actively pursue the player and utilize their full movement and attack capabilities.

Future work would focus on tighter integration between these distinct systems, such as allowing the RL agents' difficulty to influence the GenAI narrative, or using the PCG layout to constrain agent training environments dynamically. Nevertheless, this project successfully demonstrates that modern AI techniques can be orchestrated to build a game loop that is both structurally sound and creatively varied.


\bibliographystyle{model1-num-names}
\bibliography{references}

\newpage
\appendix
\section{Tasks Allocation}
\label{app:contrib}
The following is a breakdown of the main contributions made by each team member throughout the project:
\begin{itemize}
  \item \textbf{Oriol Miró}: Procedural dungeon generation (BSP implementation)
  \item \textbf{Dániel Mácsai}: Procedural room population, chapter theming (CA implementation)
  \item \textbf{Jean Dié}: Initial project setup, open-world prototype with basic procedural generation, and generative AI pipeline (narrative, music, and voice synthesis)
  \item \textbf{Bruno Sánchez}: Boss agents (Reinforcement Learning design and training)
\end{itemize}

\section{PCG Pseudocode}
\label{app:pcg_pseudocode}

\begin{lstlisting}[basicstyle=\ttfamily\small, frame=single]
Algorithm InitRun(runSeed, totalChapters):
  rng <- Random(runSeed)
  for k in 0..totalChapters-1:
    chapter[k].seed <- rng.NextInt()
    chapter[k].completed <- false
  checkpoint <- -1
  LoadChapter(0)

Algorithm LoadChapter(k):
  current <- k
  WorldGenerator.SetThemeForNextGeneration(chapterThemes[current])
  WorldGenerator.GenerateWithSeedAndPlacePlayer(chapter[current].seed)
  PlayerHealth.Reset()
  if current > 0: IntroductionDialogue.OnChapterEntered(current)

Algorithm OnBossDefeated():
  chapter[current].completed <- true
  checkpoint <- max(checkpoint, current)
  if current == totalChapters-1:
    TransitionUI.ShowFinalVictory(); return
  TransitionUI.ShowBossDefeatedText()
  WorldGenerator.SpawnChapterExitPortalAtBossRoom()

Algorithm UseChapterExit():
  if chapter[current].completed:
    LoadChapter(current + 1)

Algorithm OnPlayerDeath():
  TransitionUI.ShowDeathScreen(current)
  if current > checkpoint:
    chapter[current].seed <- rng.NextInt()   // regenerate only current chapter
  LoadChapter(current)

Algorithm GenerateWithSeedAndPlacePlayer(seed):
  rng <- Random(seed)
  ClearDungeon(); ApplyTheme()

  leaves <- BSP_Split(mapRect, rng, minLeafSize, roomCount)
  rooms  <- CarveRooms(leaves, rng, minRoomSize, maxRoomSize)
  edges  <- CompleteGraph(rooms, weight = centerDistance)
  conn   <- MST(edges) + ExtraEdges(edges, extraConnections)
  CarveCorridors(conn, rooms, rng)

  PlaceWallsOutsideAndCameraBounds()
  PopulateRooms(rooms, rng)

  PlacePlayer(StartRoom(rooms))
  PlaceBoss(BossRoom(rooms))                // boss room stored for portal spawn
\end{lstlisting}

\section{Generative AI Pipeline Details}
\label{app:genai_details}

This appendix provides additional technical details for the generative AI pipeline described in Section~\ref{subsec:genai_pipeline}.

\paragraph{Chapter Configuration}
Each chapter's generation parameters are defined in \texttt{config.json}:

\begin{lstlisting}[language=json, basicstyle=\ttfamily\scriptsize, frame=single]
{
  "seed": 54321,
  "models": {"narrative": "mistral-nemo", "music": "facebook/musicgen-small"},
  "chapters": [
    {
      "name": "The Verdant Prison",
      "boss": "Thornback",
      "environment": "sunlit forest with ancient trees and overgrown ruins",
      "progress_context": "This is the beginning. The player just arrived.",
      "music_prompt": "calm fantasy ambient music, gentle flute and strings,
                       forest atmosphere, mysterious, 90 bpm"
    },
    {"name": "The Eternal Night", "boss": "The Shade", ...},
    {"name": "The Burning End", "boss": "The Scorcher", ...}
  ]
}
\end{lstlisting}

\paragraph{Narrative Prompt Template}
The LLM receives the following prompt structure:

\begin{lstlisting}[basicstyle=\ttfamily\scriptsize, frame=single]
You are a fantasy game narrator with a deep, dramatic voice.
Setting: {chapter_name}. The boss is {boss}.
Context: {progress_context}

Write exactly 3 dramatic narrator lines (full sentences, 15-25 words each)
that set the mood and hint at the danger ahead. Be atmospheric and mysterious.
Output ONLY the 3 numbered lines:
1.
2.
3.
\end{lstlisting}

\paragraph{Example Output (Chapter 0 - The Verdant Prison)}

\begin{lstlisting}[basicstyle=\ttfamily\scriptsize, frame=single]
{
  "roomIndex": 0,
  "environment": "sunlit forest with ancient trees and overgrown ruins",
  "npc": {
    "name": "The Narrator",
    "dialogue": [
      "In the heart of the Verdant Prison, where whispers of ancient
       sorcery linger in the moss-kissed air...",
      "...the echoes of desperate pleas bounce off the living walls,
       swallowed by the relentless hum of unseen horrors.",
      "Beware, brave soul, for Thornback's shadows grow long..."
    ]
  },
  "quest": {"objective": "Defeat Thornback to proceed", "type": "Boss"},
  "lore": {"title": "The Verdant Prison", "content": "..."}
}
\end{lstlisting}

\paragraph{Output File Structure}
The generation script produces per chapter:

\begin{lstlisting}[basicstyle=\ttfamily\scriptsize, frame=single]
StreamingAssets/GeneratedContent/
  manifest.json           # Content manifest with seed
  chapters/
    chapter_0/
      narrative.json      # NPC name and dialogue lines
      music.wav           # 30-second loopable ambient track
      voice_0.wav         # Narrated dialogue lines
      voice_1.wav
      voice_2.wav
\end{lstlisting}

%% Authors are advised to submit their bibtex database files. They are
%% requested to list a bibtex style file in the manuscript if they do
%% not want to use model1-num-names.bst.

%% References without bibTeX database:

% \begin{thebibliography}{00}

%% \bibitem must have the following form:
%%   \bibitem{key}...
%%

% \bibitem{}

% \end{thebibliography}


\end{document}
