\documentclass{beamer}

% Theme selection
\usetheme{Madrid}
\usecolortheme{beaver}

% Package inclusion
\usepackage{graphicx}
\usepackage{tikz}
\usetikzlibrary{shapes,arrows,positioning}

% Title page information
\title[Sprint 3]{Sprint 3: Final Report}
\subtitle{Procedurally Generated 2D RPG with AI-Driven Narrative}
\author[Oriol, Jean, Bruno, Dániel]{Oriol Miró \and Jean Dié \and Bruno Sánchez \and Dániel Mácsai}
\institute[NDVW]{Master in Artificial Intelligence \\ Normative and Dynamic Virtual Worlds}
\date{December 16, 2025}

\begin{document}

% --- TITLE SLIDE ---
\begin{frame}
    \titlepage
\end{frame}

% --- OUTLINE SLIDE ---
\begin{frame}{Outline}
    \tableofcontents
\end{frame}

% --- INTRODUCTION SECTION ---
\section{Introduction \& Overview}

\begin{frame}{Project Overview}
    \begin{itemize}
        \item 2D top-down RPG.
        \item Dungeon, story, and music are procedurally generated.
        \item Three chapters, each ending with a boss fight.
        \item Defeating a boss "freezes" that chapter's generated content.
    \end{itemize}
    \begin{figure}
        \includegraphics[width=\linewidth]{figures/game_flow.png}
    \end{figure}
\end{frame}

\begin{frame}{Foundation \& Our Contribution}
    \begin{block}{Starting Point: Unity Tutorial}
        We are building upon a tutorial from Udemy, which provides a solid base for:
        \begin{itemize}
            \item Player movement and animation.
            \item Tilemap-based environments.
            \item A functional weapon and combat system.
        \end{itemize}
    \end{block}
    
    \begin{block}{Our Project's Innovations}
        We extend the tutorial by adding:
        \begin{itemize}
            \item Procedural dungeon generation (BSP Algorithm).
            \item An AI-driven generative pipeline for narrative and music.
            \item Boss behaviors trained with Reinforcement Learning.
            \item The novel "freezing checkpoint" mechanic tied to the narrative.
        \end{itemize}
    \end{block}
\end{frame}

% --- TECHNICAL PROPOSAL SECTION ---
\section{AI Systems \& Proposal}

\subsection{Procedural Content Generation}

\begin{frame}{Procedural Dungeon Generation}
    \begin{block}{Binary Space Partitioning (BSP) Algorithm}
        The dungeon layout is generated in several steps:
        \begin{enumerate}
            \item \textbf{Leaf Splitting:} A large rectangle is recursively split into smaller sub-regions (leaves).
            \item \textbf{Room Carving:} A randomly sized room is placed inside each leaf.
            \item \textbf{Connectivity:} A Minimum Spanning Tree (MST) is used to connect all rooms with corridors + some extra connections to create loops.
        \end{enumerate}
    \end{block}
    \begin{columns}
    \begin{column}{0.5\textwidth}
        \begin{figure}
        \includegraphics[width=\linewidth]{figures/ex_BSP_geometry.png}
        \end{figure}
    \end{column}
    \begin{column}{0.5\textwidth}
        \begin{figure}
        \includegraphics[width=\linewidth]{figures/ex_player_POV.png}
        \end{figure}
    \end{column}
    \end{columns}
\end{frame}

\subsection{Generative AI}

\begin{frame}[fragile]{Generative AI Pipeline Architecture}
    \begin{columns}
    \begin{column}{0.6\textwidth}
    \begin{figure}
        \includegraphics[width=\linewidth]{figures/genai_pipeline.png}
    \end{figure}
    \end{column}
    \begin{column}{0.4\textwidth}
    \begin{itemize}
        \item A \textbf{Python FastAPI backend} handles all AI inference (vision, language, music).
        \item The \textbf{Unity game client} communicates with the backend via HTTP requests.
        \item A two-layer cache (server-side and client-side) is used to reduce latency.
    \end{itemize}
    \end{column}
    \end{columns}
\end{frame}

\begin{frame}{Generative AI Pipeline: The 3 Steps}
    \begin{block}{1. Vision-Based Room Analysis}
        \textbf{Model:} \texttt{moondream2} \\
        Takes a screenshot of the generated room and outputs a JSON describing its environment, atmosphere, and key features. This text seeds the next two steps.
    \end{block}
    
    \begin{block}{2. Narrative Generation}
        \textbf{Model:} \texttt{llama2} \\
        Uses the vision analysis to generate context-aware story elements: NPC names, dialogue, quest objectives, and lore entries, all formatted in JSON.
    \end{block}

    \begin{block}{3. Music Generation}
        \textbf{Model:} \texttt{MusicGen Medium} \\
        Generates loopable, ambient background music based on simple keywords (e.g., "mysterious", "tense boss fight") derived from the vision analysis.
    \end{block}
\end{frame}

\subsection{Reinforcement Learning}

\begin{frame}{Boss Agents (Reinforcement Learning)}
    \begin{columns}[t]
    \begin{column}{0.33\textwidth}
        \textbf{Observations (18)}
        \begin{itemize}\small
            \item Position \& velocity
            \item Health \& stamina
            \item Weapon info
            \item Cooldowns
            \item Aim angles
        \end{itemize}
    \end{column}
    
    \begin{column}{0.33\textwidth}
        \textbf{Actions (3+1)}
        \begin{itemize}\small
            \item Movement (5)
            \item Attack (2)
            \item Dash (2)
            \item Aim (continuous)
        \end{itemize}
    \end{column}
    
    \begin{column}{0.33\textwidth}
        \textbf{Rewards}
        \begin{itemize}\small
            \item Victory: +1.0
            \item Defeat: -0.8
            \item Hit: +0.5
            \item Damage: -0.4
            \item Proximity/time: $\pm$0.001
        \end{itemize}
    \end{column}
    \end{columns}
    
    \vspace{0.2cm}
    \begin{figure}
        \centering
        \includegraphics[width=0.5\linewidth]{figures/rl_training_arena.png}
    \end{figure}
\end{frame}


\begin{frame}{RL Training Strategy}
    \begin{figure}[h]
        \centering
        \includegraphics[width=0.65\linewidth]{figures/rl_pretraining_selfplay.png}
        \caption{Cumulative reward during the two-phase training: pre-training against heuristic opponent followed by self-play refinement.}
        \label{fig:rl_pretraining_selfplay}
    \end{figure}
    \begin{figure}[h]
        \centering
        \includegraphics[width=0.65\linewidth]{figures/rl_selfplay_comparison.png}
        \caption{Comparison of self-play training performance: pure self-play (pink) versus self-play initialized from pre-trained policy (blue).}
        \label{fig:rl_selfplay_comparison}
    \end{figure}
\end{frame}

% --- CONCLUSION SECTION ---
\section{Conclusion}

\begin{frame}{Conclusion}
    \begin{block}{Contributions}
        \begin{itemize}
            \item Core game mechanics established based on the tutorial.
            \item FSM-based AI for basic enemies is functional.
            \item Procedural dungeon generator (BSP) is implemented.
            \item The three-stage GenAI pipeline (Vision, Narrative, Music) is designed and components have been tested individually.
            \item The RL training successfully achieves human-like behavior for dynamic boss encounters.
        \end{itemize}
    \end{block}
\end{frame}

\end{document}