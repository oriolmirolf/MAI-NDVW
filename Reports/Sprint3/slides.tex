\documentclass{beamer}

% Theme selection
\usetheme{Madrid}
\usecolortheme{beaver}

% Package inclusion
\usepackage{graphicx}
\usepackage{tikz}
\usepackage{attachfile2}
\usepackage{hyperref}
\usetikzlibrary{shapes,arrows,positioning}

% Title page information
\title[Sprint 3]{Sprint 3: Final Report}
\subtitle{Procedurally Generated 2D RPG with AI-Driven Narrative}
\author[Oriol, Jean, Bruno, Dániel]{Oriol Miró \and Jean Dié \and Bruno Sánchez \and Dániel Mácsai}
\institute[NDVW]{Master in Artificial Intelligence \\ Normative and Dynamic Virtual Worlds}
\date{December 16, 2025}

\begin{document}

% --- TITLE SLIDE ---
\begin{frame}
    \titlepage
\end{frame}

% --- OUTLINE SLIDE ---
\begin{frame}{Outline}
    \tableofcontents
\end{frame}

% --- INTRODUCTION SECTION ---
\section{Introduction \& Overview}

\begin{frame}{Project Overview}
    \begin{itemize}
        \item 2D top-down RPG.
        \item Dungeon, story, and music are procedurally generated.
        \item Three chapters, each ending with a boss fight.
        \item Defeating a boss "freezes" that chapter's generated content.
    \end{itemize}
    \begin{figure}
        \includegraphics[width=\linewidth]{figures/game_flow.png}
    \end{figure}
\end{frame}

\begin{frame}{Foundation \& Our Contribution}
    \begin{block}{Starting Point: Unity Tutorial}
        We built upon a tutorial from Udemy, which provided a solid base for:
        \begin{itemize}
            \item Player movement and animation.
            \item Tilemap-based environments.
            \item A functional weapon and combat system.
        \end{itemize}
    \end{block}
    
    \begin{block}{Our Project's Innovations}
        We extend the tutorial by adding:
        \begin{itemize}
            \item Procedural dungeon generation (BSP) with CA-based room population.
            \item An AI-driven generative pipeline for narrative and music.
            \item Boss behaviors trained with Reinforcement Learning.
            \item A three-chapter narrative structure with corresponding themes.
        \end{itemize}
    \end{block}
\end{frame}

% --- TECHNICAL PROPOSAL SECTION ---
\section{AI Systems \& Proposal}

% --- REPLACE ONLY YOUR PCG SUBSECTION WITH THIS (3 slides, spaced) ---
\subsection{Procedural Content Generation}

\begin{frame}{PCG: DungeonRunManager (Chapter persistence)}

    \begin{block}{Chapter-based run logic (boss-gated checkpoints)}
        \begin{itemize}
            \item The run is split into 3 chapters; each chapter dungeon is generated from a stored seed.
            \item After defeating a chapter boss, everything the player cleared so far in that chapter stays fixed for the rest of the run.
            \item On player death, only the \emph{current unfinished} chapter is regenerated; previously cleared chapters keep their content.
        \end{itemize}
    \end{block}
\end{frame}

\begin{frame}{PCG: Connectivity Modes (Corridors vs.\ Portals)}
    \begin{block}{Why treat connectivity as a mode?}
        It directly affects navigation readability \emph{and} the visual structure of rooms (important for consistent screenshots later).
    \end{block}

    \begin{columns}[t]
    \begin{column}{0.44\textwidth}
        \begin{figure}
            \centering
            \includegraphics[width=\linewidth]{figures/corridor_example.png}
            \caption*{\scriptsize Corridor connectivity between rooms.}
        \end{figure}
    \end{column}

    \begin{column}{0.44\textwidth}
        \begin{figure}
            \centering
            \includegraphics[width=\linewidth]{figures/portal_example.png}
            \caption*{\scriptsize Portal connectivity example.}
        \end{figure}
    \end{column}
    \end{columns}
\end{frame}

% --- INSERT THIS SLIDE BETWEEN "Connectivity Modes" AND "Run Transitions" ---
\begin{frame}{PCG: Layout Generation (BSP + global connectivity)}
    \begin{block}{Why BSP here?}
        It gives fast, controllable variation: many room candidates, bounded sizes, and no overlaps---so layouts stay readable under regeneration.
    \end{block}

    \begin{block}{High-level pipeline}
        \begin{itemize}
            \item Recursively split the map rectangle into leaves (Binary Space Partitioning).
            \item Sample one room per feasible leaf (with margins and size bounds).
            \item Enforce full reachability with a Minimum Spanning Tree over room centres,
                  optionally adding a few extra edges to avoid overly linear layouts.
        \end{itemize}
    \end{block}

\end{frame}


\begin{frame}{PCG: Run Transitions (Death / Victory / Exit Portal)}
    \begin{block}{Boss-defeat gating}
        Progression is enforced by spawning the chapter exit portal \emph{only after} the boss is defeated, preventing accidental skips and making success explicit.
    \end{block}

    \begin{figure}
      \centering
      \begin{minipage}[t]{0.32\linewidth}
        \centering
        \includegraphics[width=\linewidth]{figures/death_screen.png}\\
        \scriptsize \textbf{(a)} Death screen.
      \end{minipage}
      \hfill
      \begin{minipage}[t]{0.32\linewidth}
        \centering
        \includegraphics[width=\linewidth]{figures/final_victory.png}\\
        \scriptsize \textbf{(b)} Final victory.
      \end{minipage}
      \hfill
      \begin{minipage}[t]{0.32\linewidth}
        \centering
        \includegraphics[width=\linewidth]{figures/boss_portal.png}\\
        \scriptsize \textbf{(c)} Exit portal after boss defeat.
      \end{minipage}

      \caption{\scriptsize Transitions controlled by \texttt{DungeonRunManager}: death regenerates only the current unfinished chapter; boss defeat unlocks the exit; final victory ends the run.}
    \end{figure}
\end{frame}


\begin{frame}{CA-Based Room Population}
    \begin{block}{Motivation}
        Rooms generated by BSP need to be filled with content. We use \textbf{Cellular Automata (CA)} to create natural-looking features instead of uniform or grid-based placement.
    \end{block}

    \begin{block}{Three CA Applications}
        \begin{itemize}
            \item \textbf{Lakes:} Natural water bodies with higher survival thresholds 
            \item \textbf{Paths:} Portal-seeded CA where paths emerge from room entrances
            \item \textbf{Decorations:} Clustered small bushes for visual variety
        \end{itemize}
    \end{block}

    \begin{block}{Large Obstacles}
        Trees and buildings use randomized placement with collision detection (not CA).
    \end{block}
\end{frame}

\begin{frame}{CA Algorithm: Core Idea}
    \textbf{How it works:}
    \begin{enumerate}
        \item Randomly seed cells with initial density
        \item For each iteration:
        \begin{itemize}
            \item Count neighbors (Moore neighborhood: 8 cells)
            \item Cells with $\geq$ surviveMin neighbors stay alive
            \item Dead cells with $\geq$ birthMin neighbors become alive
        \end{itemize}
        \item Remove orphan tiles (cleanup)
    \end{enumerate}
\end{frame}

\begin{frame}{Chapter Theme System}
    \begin{block}{Architecture}
        Each chapter theme (ScriptableObject) contains:
        \begin{itemize}
            \item Visual assets (tiles, prefabs)
            \item CA parameters (density, iterations, survival/birth rules)
            \item Atmosphere settings (lighting, colors)
        \end{itemize}
    \end{block}

    \begin{columns}
    \begin{column}{0.33\textwidth}
        \centering
        \includegraphics[width=\linewidth,height=4cm,keepaspectratio]{figures/ca_screenshots/forest_theme.png}

        {\small Forest (Chapter 1)}
    \end{column}
    \begin{column}{0.33\textwidth}
        \centering
        \includegraphics[width=\linewidth,height=4cm,keepaspectratio]{figures/ca_screenshots/night_theme.png}

        {\small Night (Chapter 2)}
    \end{column}
    \begin{column}{0.33\textwidth}
        \centering
        \includegraphics[width=\linewidth,height=4cm,keepaspectratio]{figures/ca_screenshots/hell_theme.png}

        {\small Hell (Chapter 3)}
    \end{column}
    \end{columns}
\end{frame}

\begin{frame}{CA Applications: Examples}
    \begin{columns}
    \begin{column}{0.5\textwidth}
        \centering
        \textbf{Lakes with Borders}

        \includegraphics[width=\linewidth,height=5cm,keepaspectratio]{figures/ca_screenshots/lake_example.png}

        {Higher survival threshold}
    \end{column}
    \begin{column}{0.5\textwidth}
        \centering
        \textbf{Portal-Seeded Paths}

        \includegraphics[width=\linewidth,height=5cm,keepaspectratio]{figures/ca_screenshots/paths_example.png}

        {\small High density near portals}
    \end{column}
    \end{columns}
\end{frame}


\subsection{Generative AI}

\begin{frame}{Generative AI -- Pre-Generation Phase}
    \begin{figure}
        \centering
        \includegraphics[width=0.95\linewidth]{figures/genai_pregen.png}
    \end{figure}
    \vspace{0.2cm}
    \small
    \textbf{Models:} mistral-nemo (narrative), MusicGen Small (music), XTTS v2 (voice cloning)
\end{frame}

\begin{frame}{Generative AI -- Runtime Phase (Unity)}
    \begin{figure}
        \centering
        \includegraphics[width=0.95\linewidth]{figures/genai_runtime.png}
    \end{figure}
    \vspace{0.2cm}
    \small
    Pre-generated content loaded at runtime -- \textbf{zero latency} for players
\end{frame}

\begin{frame}{Generated Content Samples}
    \begin{block}{Voice Cloning with XTTS v2}
        \textbf{Reference sample (15s):} \href{https://www.youtube.com/watch?v=ErrcHBlxOBs}{\textcolor{blue}{\underline{BBC Documentary Narration}}} \\[0.2cm]
        \textbf{Generated voice:} \textattachfile[color=0 0 1, icon=Speaker]{audio/chapter0_voice.wav}{\textcolor{blue}{Play Audio}}
    \end{block}

    \begin{block}{Generated Narrator Dialogue}
        \small\textit{``In the heart of the Verdant Prison, where whispers of ancient sorcery linger in the moss-kissed air...''}
    \end{block}

    \begin{block}{Music Generation (MusicGen Small)}
        \textbf{Chapter 0 -- Forest theme:} calm fantasy ambient, flute and strings, 90 bpm \\[0.2cm]
        \textattachfile[color=0 0 1, icon=Speaker]{audio/chapter0_music.wav}{\textcolor{blue}{Play Audio}}
    \end{block}
\end{frame}

\begin{frame}{GenAI: Future Perspectives}
    \begin{block}{Proof of Concept}
        This pipeline demonstrates how generative AI can create game content that would traditionally require manual creation.
    \end{block}

    \begin{columns}[t]
    \begin{column}{0.5\textwidth}
    \textbf{Near-term: Asset Generation}
    \begin{itemize}
        \item Indie developers generating voice/music
        \item Reduces production costs
        \item Democratizes game development
    \end{itemize}
    \end{column}

    \begin{column}{0.5\textwidth}
    \textbf{Long-term: AI-Native Engine}
    \begin{itemize}
        \item Game engine built around GenAI
        \item Fully customizable experiences
        \item Agent-based NPCs with emergent behavior
    \end{itemize}
    \end{column}
    \end{columns}
\end{frame}

\subsection{Reinforcement Learning}

\begin{frame}{Boss Agents (Reinforcement Learning)}
    \begin{columns}[t]
    \begin{column}{0.33\textwidth}
        \textbf{Observations (18)}
        \begin{itemize}\small
            \item Position \& velocity
            \item Health \& stamina
            \item Weapon info
            \item Cooldowns
            \item Aim angles
        \end{itemize}
    \end{column}
    
    \begin{column}{0.33\textwidth}
        \textbf{Actions (3+1)}
        \begin{itemize}\small
            \item Movement (5)
            \item Attack (2)
            \item Dash (2)
            \item Aim (continuous)
        \end{itemize}
    \end{column}
    
    \begin{column}{0.33\textwidth}
        \textbf{Rewards}
        \begin{itemize}\small
            \item Victory: +1.0
            \item Defeat: -0.8
            \item Hit: +0.5
            \item Damage: -0.4
            \item Proximity/time: $\pm$0.001
        \end{itemize}
    \end{column}
    \end{columns}
    
    \vspace{0.2cm}
    \begin{figure}
        \centering
        \includegraphics[width=0.5\linewidth]{figures/rl_training_arena.png}
    \end{figure}
\end{frame}


\begin{frame}{RL Training Strategy}
    \begin{figure}[h]
        \centering
        \includegraphics[width=0.65\linewidth]{figures/rl_pretraining_selfplay.png}
        \caption{Cumulative reward during the two-phase training: pre-training against heuristic opponent followed by self-play refinement.}
        \label{fig:rl_pretraining_selfplay}
    \end{figure}
    \begin{figure}[h]
        \centering
        \includegraphics[width=0.65\linewidth]{figures/rl_selfplay_comparison.png}
        \caption{Comparison of self-play training performance: pure self-play (pink) versus self-play initialized from pre-trained policy (blue).}
        \label{fig:rl_selfplay_comparison}
    \end{figure}
\end{frame}

% --- CONCLUSION SECTION ---
\section{Conclusion}

\begin{frame}{Conclusion}
    \begin{block}{Contributions}
        \begin{itemize}
            \item Core game mechanics established based on the tutorial.
            \item FSM-based AI for basic enemies is functional.
            \item Procedural dungeon generator (BSP) and room population (CA) are implemented.
            \item The three-stage GenAI pipeline (Vision, Narrative, Music) is designed and components have been tested individually.
            \item The RL training successfully achieves human-like behavior for dynamic boss encounters.
        \end{itemize}
    \end{block}
\end{frame}

\end{document}