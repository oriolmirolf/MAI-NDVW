\documentclass{beamer}

% Theme selection
\usetheme{Madrid}
\usecolortheme{beaver}

% Package inclusion
\usepackage{graphicx}
\usepackage{tikz}
\usetikzlibrary{shapes,arrows,positioning}

% Title page information
\title[Sprint 2]{Sprint 2: Project Progress}
\subtitle{Procedurally Generated 2D RPG with AI-Driven Narrative}
\author[Oriol, Jean, Bruno, Dániel]{Oriol Miró \and Jean Dié \and Bruno Sánchez \and Dániel Mácsai}
\institute[NDVW]{Master in Artificial Intelligence \\ Normative and Dynamic Virtual Worlds}
\date{November 18, 2025}

\begin{document}

% --- TITLE SLIDE ---
\begin{frame}
    \titlepage
\end{frame}

% --- OUTLINE SLIDE ---
\begin{frame}{Outline}
    \tableofcontents
\end{frame}

% --- INTRODUCTION SECTION ---
\section{Introduction \& Overview}

\begin{frame}{Project Overview}
    \begin{itemize}
        \item A 2D top-down RPG where the dungeon
        \item Story, and music are procedurally generated.
        \item Three chapters, each ending with a boss fight.
        \item Defeating a boss "freezes" that chapter's generated content
    \end{itemize}
    \begin{figure}
        \includegraphics[width=\linewidth]{figures/game_flow.png}
    \end{figure}
\end{frame}

\begin{frame}{Foundation \& Our Contribution}
    \begin{block}{Starting Point: Unity Tutorial}
        We are building upon a tutorial from Udemy, which provides a solid base for:
        \begin{itemize}
            \item Player movement and animation.
            \item Tilemap-based environments.
            \item A functional weapon and combat system.
        \end{itemize}
    \end{block}
    
    \begin{block}{Our Project's Innovations}
        We extend the tutorial by adding:
        \begin{itemize}
            \item Procedural dungeon generation (BSP Algorithm).
            \item An AI-driven generative pipeline for narrative and music.
            \item Boss behaviors trained with Reinforcement Learning.
            \item The novel "freezing checkpoint" mechanic tied to the narrative.
        \end{itemize}
    \end{block}
\end{frame}

% --- GAME AGENTS SECTION ---
\section{Game Agents}

\begin{frame}{Player Agent}
    \begin{columns}
        \begin{column}{0.5\textwidth}
            \textbf{Movement \& Abilities}
            \begin{itemize}
                \item WASD for movement, mouse for 360° aiming.
                \item A dash ability for invulnerability, consuming a regenerating stamina bar.
            \end{itemize}
            
            \vspace{1cm}
            
            \textbf{Health \& Resources}
            \begin{itemize}
                \item Health replenished by collecting hearts.
                \item Enemies drop coins (for a potential future shop system).
            \end{itemize}
        \end{column}
        \begin{column}{0.5\textwidth}
            \textbf{Weapons}
            \begin{itemize}
                \item \textbf{Sword:} Melee sweep attack.
                \item \textbf{Bow:} Long-range single-target arrow.
                \item \textbf{Magic Staff:} Medium-range piercing beam.
            \end{itemize}
            \begin{center}
            \includegraphics[width=\linewidth]{figures/example_from_tutorial.png}
            \end{center}
        \end{column}
    \end{columns}
\end{frame}

\begin{frame}{Basic Monster Agents}
    \begin{columns}
        \begin{column}{0.5\textwidth}
            \textbf{Behavior: Finite State Machine}
            \begin{itemize}
                \item \textbf{Roaming State:} Moves to random points.
                \item \textbf{Attacking State:} Engages the player when they enter a defined range.
            \end{itemize}
            \begin{figure}
                \includegraphics[width=\linewidth]{figures/EnemyAI FSM.png}
            \end{figure}
        \end{column}
        \begin{column}{0.5\textwidth}
            \textbf{Enemy Types}
            \begin{itemize}
                \item \textbf{Slimes:} Simple melee enemies that deal damage on contact.
                \item \textbf{Ghosts:} Ranged shooters that fire projectiles in various patterns.
            \end{itemize}
            \begin{figure}
                \includegraphics[height=1.5cm]{figures/Slime_Sheet.png}
                \includegraphics[height=1.5cm]{figures/Ghost.png}
            \end{figure}
        \end{column}
    \end{columns}
\end{frame}

\begin{frame}{Boss Agents}
    \begin{itemize}
        \item Behaviors are controlled by agents trained via RL $\rightarrow$  dynamic, non-scripted encounters.
        \item Bosses use the same abilities as the player but are controlled by a trained RL agent.
        \item  Different training checkpoints of the same agent to create the three bosses.
    \end{itemize}
    \begin{figure}
        \centering
        \includegraphics[width=0.6\linewidth]{figures/rl_training_arena.png}
    \end{figure}
\end{frame}

% --- NARRATIVE SECTION ---
\section{Narrative}

\begin{frame}{The Scenario: The Chronicle of Light}
    \begin{itemize}
        \item \textbf{Protagonist:} Sir Cael, a knight trapped within a living book, the \textit{Chronicle of Light}.
        \item \textbf{The Curse:} Every time Sir Cael dies, the world (dungeon, characters, etc.) is rewritten by the Chronicler.
        \item \textbf{The Goal:} To find and defeat the three anchors of the story: the Sentinel, the Herald, and the Chronicler himself.
        \item \textbf{Narrative-Mechanic Link:} When a boss is defeated, its chapter is "fixed in ink." The dungeon layout and story for that section no longer change upon death, creating a permanent checkpoint.
    \end{itemize}
\end{frame}

% --- TECHNICAL PROPOSAL SECTION ---
\section{AI Systems \& Proposal}

\begin{frame}{Procedural Dungeon Generation}
    \begin{block}{Binary Space Partitioning (BSP) Algorithm}
        The dungeon layout is generated in several steps:
        \begin{enumerate}
            \item \textbf{Leaf Splitting:} A large rectangle is recursively split into smaller sub-regions (leaves).
            \item \textbf{Room Carving:} A randomly sized room is placed inside each leaf.
            \item \textbf{Connectivity:} A Minimum Spanning Tree (MST) is used to connect all rooms with corridors + some extra connections to create loops.
        \end{enumerate}
    \end{block}
    \begin{columns}
    \begin{column}{0.5\textwidth}
        \begin{figure}
        \includegraphics[width=\linewidth]{figures/ex_BSP_geometry.png}
        \end{figure}
    \end{column}
    \begin{column}{0.5\textwidth}
        \begin{figure}
        \includegraphics[width=\linewidth]{figures/ex_player_POV.png}
        \end{figure}
    \end{column}
    \end{columns}
\end{frame}

\begin{frame}[fragile]{Generative AI Pipeline Architecture}
    \begin{columns}
    \begin{column}{0.6\textwidth}
    \begin{figure}
        \includegraphics[width=\linewidth]{figures/genai_pipeline.png}
    \end{figure}
    \end{column}
    \begin{column}{0.4\textwidth}
    \begin{itemize}
        \item A \textbf{Python FastAPI backend} handles all AI inference (vision, language, music).
        \item The \textbf{Unity game client} communicates with the backend via HTTP requests.
        \item A two-layer cache (server-side and client-side) is used to reduce latency.
    \end{itemize}
    \end{column}
    \end{columns}
\end{frame}

\begin{frame}{Generative AI Pipeline: The 3 Steps}
    \begin{block}{1. Vision-Based Room Analysis}
        \textbf{Model:} \texttt{moondream2} \\
        Takes a screenshot of the generated room and outputs a JSON describing its environment, atmosphere, and key features. This text seeds the next two steps.
    \end{block}
    
    \begin{block}{2. Narrative Generation}
        \textbf{Model:} \texttt{llama2} \\
        Uses the vision analysis to generate context-aware story elements: NPC names, dialogue, quest objectives, and lore entries, all formatted in JSON.
    \end{block}

    \begin{block}{3. Music Generation}
        \textbf{Model:} \texttt{MusicGen Medium} \\
        Generates loopable, ambient background music based on simple keywords (e.g., "mysterious", "tense boss fight") derived from the vision analysis.
    \end{block}
\end{frame}


\begin{frame}{Reinforcement Learning for Bosses}
    \begin{block}{Training Setup (Unity ML-Agents)}
        \begin{itemize}
            \item \textbf{Algorithm:} Proximal Policy Optimization (PPO).
            \item \textbf{Environment:} A custom "Arena" for self-play, where the agent learns by fighting a copy of itself.
            \item \textbf{Observation Space:} Agent/opponent position and health.
            \item \textbf{Action Space:} Movement, dashing, and choice of three weapon attacks.
            \item \textbf{Reward Function:} Rewards hitting the opponent and winning; penalizes taking damage, losing, and taking too long.
        \end{itemize}
    \end{block}
    \begin{alertblock}{Goal}
        To create bosses that can react to the player's actions and develop complex, human-like strategies without being explicitly scripted.
    \end{alertblock}
\end{frame}

% --- CONCLUSION SECTION ---
\section{Conclusion \& Next Steps}

\begin{frame}{Progress Summary \& Next Steps}
    \begin{block}{Progress This Sprint}
        \begin{itemize}
            \item Core game mechanics established based on the tutorial.
            \item FSM-based AI for basic enemies is functional.
            \item Procedural dungeon generator (BSP) is implemented.
            \item The three-stage GenAI pipeline (Vision, Narrative, Music) is designed and components have been tested individually.
            \item The RL training environment for bosses is set up.
        \end{itemize}
    \end{block}
    
    \begin{block}{Goals for Sprint 3}
        \begin{itemize}
            \item Fully integrate the BSP generator with the Generative AI pipeline.
            \item Train the boss agents and integrate them into the game.
            \item Implement the chapter "freezing" and story-driven room population.
        \end{itemize}
    \end{block}
\end{frame}

\end{document}