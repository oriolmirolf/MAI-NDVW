\documentclass{beamer}
\usetheme{Madrid}
\usecolortheme{whale}
\usepackage{graphicx}
\usepackage{booktabs}
\usepackage{amsmath}
\usepackage{tikz}
\usepackage{tabularx}
\usepackage[table,xcdraw]{xcolor}


\definecolor{ubburgundy}{RGB}{160, 0, 0}
\setbeamercolor{structure}{fg=ubburgundy}

\makeatletter
\setbeamertemplate{footline}{
  \leavevmode%
  \hbox{%
    \begin{beamercolorbox}[wd=.5\paperwidth,ht=2.5ex,dp=1.125ex,leftskip=.3cm]{title in head/foot}%
      \usebeamerfont{title in head/foot}\insertshorttitle
    \end{beamercolorbox}%
    \begin{beamercolorbox}[wd=.5\paperwidth,ht=2.5ex,dp=1.125ex,rightskip=.3cm plus1fil]{date in head/foot}%
      \usebeamerfont{date in head/foot}\hfill\insertframenumber{} / \inserttotalframenumber
    \end{beamercolorbox}%
  }%
}
\makeatother

\setbeamertemplate{navigation symbols}{}
\logo{}

\title{Sprint 1: Project Proposal}
\subtitle{Procedurally Generated 2D Roguelike RPG}
\author{Oriol Miró, Jean Dié, Bruno Sánchez, Dániel Mácsai}
\institute{
    \includegraphics[width=0.4\textwidth]{figures/ub_logo.png}\\[1ex]
    University of Barcelona\\
    \textit{Normative and Dynamic Virtual Worlds}
}
\date{October 7th, 2025}

\begin{document}

\begin{frame}
    \titlepage
\end{frame}

% 8-minute presentation structure:
% Title: 30s
% Game Concept: 1.5 min
% Core AI: Procedural Generation: 2.5 min
% Implementation Plan: 1.5 min
% Demo/Visuals: 1.5 min
% Q&A buffer: 30s

\section{Game Concept}

\begin{frame}{What We're Building}
    \begin{columns}[T]
        \begin{column}{0.48\textwidth}
            \textbf{The Vision}:
            \begin{itemize}
                \item Top-down 2D roguelike RPG with \textcolor{ubburgundy}{infinite replayability}
                \item Every playthrough = unique dungeon
                \item Checkpoint-based progression
            \end{itemize}

            \vspace{0.5cm}
            \textbf{Core Challenge}:
            \begin{itemize}
                \item Algorithmically generate dungeons that are:
                \begin{itemize}
                    \item Always playable
                    \item Fair and balanced
                    \item Meaningfully different
                \end{itemize}
            \end{itemize}
        \end{column}

        \begin{column}{0.48\textwidth}
            \textbf{Inspiration}:
            \begin{itemize}
                \item \textit{Enter the Gungeon}
                \item \textit{Spelunky 1 \& 2}
                \item \textit{Binding of Isaac}
                \item \textit{Dead Cells}
                \item \textit{Hades}
            \end{itemize}

            \vspace{0.3cm}
            \textbf{Why Roguelikes?}
            \begin{itemize}
                \item Perfect testbed for PCG
                \item Each run validates algorithms
                \item High replay value
                \item No manual level design needed
            \end{itemize}
        \end{column}
    \end{columns}
\end{frame}

\begin{frame}{Game Flow \& Structure}
    \begin{center}
        \textbf{Three-Stage Journey with Checkpoints}
    \end{center}

    \begin{columns}[T]
        \begin{column}{0.48\textwidth}
            \textbf{Progression System}:
            \begin{itemize}
                \item Three dungeon sections
                \item Each dungeon culminates in a boss fight
                \item Checkpoints after defeating each boss
            \end{itemize}

            \vspace{0.35cm}
            \textbf{Death Mechanic}:
            \begin{itemize}
                \item Return to last checkpoint
                \item \textcolor{ubburgundy}{Current dungeon regenerates completely}
                \item New layout, new challenges
            \end{itemize}
        \end{column}

        \begin{column}{0.48\textwidth}
            \textbf{Visual Variety}:
            \begin{itemize}
                \item Each section has a distinct theme
                \begin{enumerate}
                    \item Volcanic chambers
                    \item Overgrown ruins
                    \item Crystalline caverns
                \end{enumerate}
                \item Different layout styles
            \end{itemize}

            \vspace{0.2cm}
            \textbf{Design Philosophy}:
            \begin{itemize}
                \item Respect player's time
                \item Permanent progress via checkpoints
                \item Endless variety via regeneration
            \end{itemize}
        \end{column}
    \end{columns}
\end{frame}

\section{Core AI: Procedural Generation}

\begin{frame}{Procedural Content Generation (PCG)}
    \begin{center}
        \textbf{The Heart of Our Project}
    \end{center}

    \begin{columns}[T]
        \begin{column}{0.48\textwidth}
            \textbf{Algorithm 1: Cellular Automata}
            \begin{itemize}
                \item Creates organic, cave-like dungeons
                \item Rule-based iteration
                \item Natural-looking environments
            \end{itemize}

            \vspace{0.6cm}
            \textbf{How it works}:
            \begin{enumerate}\small
                \item Start with random noise
                \item Each cell checks neighbors
                \item Apply rules: wall or floor
                \item Repeat until stable
            \end{enumerate}
        \end{column}

        \begin{column}{0.48\textwidth}
            \textbf{Algorithm 2: Binary Space Partitioning}
            \begin{itemize}
                \item Structured room-and-corridor layouts
                \item Recursive space division
                \item Architectural feel
            \end{itemize}

            \vspace{0.2cm}
            \textbf{How it works}:
            \begin{enumerate}\small
                \item Recursively split space
                \item Place rooms in leaf nodes
                \item Connect with corridors
                \item Ensures connectivity
            \end{enumerate}
        \end{column}
    \end{columns}

    \vspace{0.3cm}
    \begin{center}
        \textcolor{ubburgundy}{\textbf{Two algorithms = Variety in dungeon styles}}
    \end{center}
\end{frame}

\begin{frame}{Validation \& Quality Control}
    \begin{columns}[T]
        \begin{column}{0.48\textwidth}
            \textbf{The Challenge}:
            \begin{itemize}
                \item Random $\neq$ Playable
                \item Need intelligent validation
                \item Ensure fair gameplay
            \end{itemize}

            \vspace{0.3cm}
            \textbf{Connectivity Check}:
            \begin{itemize}
                \item All rooms reachable?
                \item Pathfinding validation
                \item No isolated areas
            \end{itemize}
        \end{column}

        \begin{column}{0.48\textwidth}
            \textbf{Pacing Validation}:
            \begin{itemize}
                \item Appropriate enemy spacing
                \item Difficulty curve
                \item Not too easy/hard
            \end{itemize}

            \vspace{0.3cm}
            \textbf{Fairness Check}:
            \begin{itemize}
                \item Room for player movement
                \item Avoidable enemy encounters
                \item No impossible situations
            \end{itemize}
        \end{column}
    \end{columns}

    \vspace{0.4cm}
    \begin{beamercolorbox}[rounded=true,shadow=true,wd=\textwidth,center]{block body}
        \textbf{Validation transforms random generation into reliable, fun gameplay}
    \end{beamercolorbox}
\end{frame}

\begin{frame}{Experimental Feature: Dynamic Narrative Generation}
    \begin{center}
        \textbf{LLM-Driven Lore and Storytelling}
    \end{center}

    \begin{columns}[T]
        \begin{column}{0.48\textwidth}
            \textbf{The Idea}:
            \begin{itemize}
                \item LLM-generated contextual story snippets
                \item Integrated through NPC dialogues or collectible items
            \end{itemize}

            \vspace{0.3cm}
            \textbf{Inputs}:
            \begin{itemize}
                \item Current game state:
                \begin{itemize}
                    \item Progress
                    \item Bosses defeated
                    \item Area theme
                \end{itemize}
            \end{itemize}
        \end{column}

        \begin{column}{0.48\textwidth}
            \textbf{Simple Approach}:
            \begin{itemize}
                \item Generate multiple options
                \item Select the most coherent
                \item Bind narrative state to checkpoints
            \end{itemize}

            \vspace{0.3cm}
            \textbf{Scope}:
            \begin{itemize}
                \item Remains \textbf{secondary} to the primary goal $\rightarrow$ \textcolor{ubburgundy}{Robust procedural dungeon generation}
            \end{itemize}
        \end{column}
    \end{columns}
\end{frame}

\section{Implementation Plan}

\begin{frame}{Development Strategy}
    \begin{columns}[T]
        \begin{column}{0.48\textwidth}
            \textbf{Technology Stack}:
            \begin{itemize}
                \item \textcolor{ubburgundy}{Unity 2D}
                \item C\# for implementation
                \item Focus on PCG algorithms
            \end{itemize}
        \end{column}

        \begin{column}{0.48\textwidth}
            \textbf{Team Structure}:
            \begin{itemize}
                \item \textbf{Manager}: Oriol Miró
                \item \textbf{AI Designer}: Dániel Mácsai
                \item \textbf{AI Tech}: Jean Dié, Bruno Sánchez
            \end{itemize}
        \end{column}
    \end{columns}

    \vspace{0.4cm}
    \begin{beamercolorbox}[rounded=true,shadow=true,wd=\textwidth,center]{block body}
        \textbf{Primary Goal}: Robust procedural generation that creates\\
        engaging, fair, and varied dungeons every time
    \end{beamercolorbox}
    
    \vspace{0.25cm}
    \begin{beamercolorbox}[rounded=true,shadow=true,wd=\textwidth,center]{block body}
        \textbf{Secondary (Optional) Goal}: LLM-driven dynamic narrative\\ elements to enhance immersion
    \end{beamercolorbox}
\end{frame}


\begin{frame}{Conclusion}
    \begin{itemize}
        \item Building a \textbf{2D roguelike RPG} powered by procedural generation
        \item \textbf{Core challenge}: Algorithmically generating playable, varied dungeons
        \item \textbf{Two algorithms}: Cellular Automata + Binary Space Partitioning
        \item \textbf{Validation systems} ensure quality and fairness
    \end{itemize}

    \vspace{0.3cm}

    \begin{beamercolorbox}[rounded=true,shadow=true,wd=\textwidth,center]{block body}
        \textbf{Demonstrating how AI-driven procedural content generation\\
        creates infinite replayability while maintaining quality}
    \end{beamercolorbox}
\end{frame}

\begin{frame}{Q \& A}
    \centering
    \Large{\textbf{Thank you for your attention!}}

    \vspace{0.5cm}
    \Large{Any questions?}
\end{frame}

\end{document}
